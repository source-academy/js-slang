% Created 2020-03-25 Wed 10:07
% Intended LaTeX compiler: pdflatex
\input source_header.tex
\newcommand{\qed}{$\Box$}
\newcommand{\Rule}[2]{\genfrac{}{}{0.7pt}{}{{\setlength{\fboxrule}{0pt}\setlength{\fboxsep}{3mm}\fbox{$#1$}}}{{\setlength{\fboxrule}{0pt}\setlength{\fboxsep}{3mm}\fbox{$#2$}}}}
\newcommand{\Rulee}[3]{\genfrac{}{}{0.7pt}{}{{\setlength{\fboxrule}{0pt}\setlength{\fboxsep}{3mm}\fbox{$#1$}}}{{\setlength{\fboxrule}{0pt}\setlength{\fboxsep}{3mm}\fbox{$#2$}}}[#3]}
\newcommand{\transition}{\rightrightarrows_s}
\newcommand{\translate}{\twoheadrightarrow}
\newcommand{\translateaux}{\hookrightarrow}
\author{koo}
\date{\today}
\title{}
\hypersetup{
 pdfauthor={koo},
 pdftitle={},
 pdfkeywords={},
 pdfsubject={},
 pdfcreator={Emacs 26.3 (Org mode 9.2.6)},
 pdflang={English}}

\renewcommand{\docheader}[3]{%

  \thispagestyle{empty}

\markright{SICP, JavaScript Adaptation, #2 #3, #1}
\begin{center}
  {\Large {\bf Specification of #2 #3}---#1 edition}\\[10mm]

  {\large Jonathan Chan Wai Hon and Zhengqun Koo}\\[5mm]

  {\large National University of Singapore \\
          School of Computing}\\[10mm]

  {\large \today}\\[10mm]
\end{center}
}
\begin{document}
	%%%%%%%%%%%%%%%%%%%%%%%%%%%%%%%%%%%%%%%%%%%%%%%
	\docheader{2021}{Source}{\S 3.4}
	%%%%%%%%%%%%%%%%%%%%%%%%%%%%%%%%%%%%%%%%%%%%%%%

\input source_intro.tex

\section{Specification of Concurrent Virtual Machine}
\label{sec:orgc6ebafc}
Compared to Source \(\S 3\), this concurrent system has the following changes:
\begin{itemize}
\item The addition of three primitive concurrent functions.
\end{itemize}
For details, see Section ``Concurrency Support'' below.

\subsection{Overview}
\label{sec:org3a177cd}
This concurrent system consists of concurrently executing (potentially multiple) code in multiple threads. Communication between threads is achieved by updating the values of memory shared between threads. Each thread is a collection of registers \(\textit{os}, \textit{pc}, \textit{e}, \textit{rs}\), and at the expiration of the time quantum, the concurrent system randomly chooses a thread to execute.

To comply with the textbook, the thread that calls \texttt{concurrent\_execute} also continues to execute the rest of its code concurrently with the threads. Furthermore, there is no \texttt{join} primitive concurrent function.

However, this design departs from the textbook, as the system does not execute parallel threads. Instead, it executes concurrent threads.

\subsection{Concurrency Support}
\label{sec:orgaa84fb2}
The following concurrent functions are supported:
\begin{itemize}
\item \texttt{concurrent\_execute}: \textit{primitive}, executes multiple threads concurrently. Each \(f_1, \cdots ,f_n\) is a nullary function that returns \texttt{undefined}, and each thread is executed by running the code in the body of the nullary function. Returns \texttt{undefined}.
\item \texttt{test\_and\_set}: \textit{primitive}, assumes the head of array \(a\) is a boolean \(b\). Sets the head of \(a\) to \texttt{true}. Returns \(b\).
\item \texttt{clear}: \textit{primitive}, sets the head of array \(a\) to \texttt{false}. Returns \texttt{undefined}.
\end{itemize}

\subsection{\texttt{EXECUTE} Rules}
\label{sec:org9d11a1b}

\subsubsection{Notes}
\label{sec:org7100746}
\begin{itemize}
\item For simplicity, heap is not represented in the rules.
\end{itemize}

\subsubsection{Compiling}
\label{sec:orgbdcf42b}
\(\Rule{E_1 \translateaux s_1 \qquad \cdots \qquad E_n \translateaux s_n}{\texttt{concurrent\_execute}(E_1, \cdots , E_n) \translateaux s_1. \cdots .s_n.\texttt{EXECUTE n}}\\\)
Each of \(s_1. \cdots .s_n\) is a string of VM instruction that loads a closure onto the operand stack.

\subsubsection{Running}
\label{sec:orgd7da3c4}
There are additional structures in our VM:
\begin{enumerate}
\item \(\textit{tq}\), a register which is a queue of threads.
\item \(\textit{to}\), a register initialized with \(0\), that indicates how many instructions are left for a thread to run.
\end{enumerate}
The state representing our VM will have two more corresponding structures:
\((\textit{os}, \textit{pc}, \textit{e}, \textit{rs}, \textit{tq}, \textit{to})\\\)
The initial state of our VM is a \(\textit{pc}\), which has empty \(\textit{os}\), \(\textit{env}\), \(\textit{rs}\), and \(\textit{tq}\), and zero timeout:
\((\langle \rangle, \textit{pc}, \langle \rangle, \langle \rangle, \langle \rangle, 0)\\\)

Some notes on program execution:
\begin{enumerate}
\item When the program calls \texttt{concurrent\_execute}, it executes the rest of its code along with the concurrent threads, therefore the sequential program becomes a concurrent thread.
\item Consider the case when a program does not use any primitive concurrent functions. To avoid distinguishing between the case when the program is sequential and the program is concurrent, therefore simplifying the rules, the sequential program must still execute like a concurrent thread. This means sequential execution may time out, and may be pushed and popped from the thread queue.
\item As a result, all programs here are concurrent programs, even if they do not call \texttt{concurrent\_execute}.
\item None of the programs here return a result, because concurrent programs should not return a result, since concurrent threads return nothing.
\item To simulate high interleaving of concurrent threads, all dequeues from \(\textit{tq}\) is done randomly: a thread is popped from a random index in \(\textit{tq}\).
\end{enumerate}

\paragraph{Thread timeout}
\label{sec:org5744804}
\(\\\Rule{}{(\textit{os}_1, \textit{pc}_1, \textit{e}_1, \textit{rs}_1, (\textit{os}_2, \textit{pc}_2, \textit{e}_2, \textit{rs}_2).\textit{tq}, 0) \transition (\textit{os}_2, \textit{pc}_2, \textit{e}_2, \textit{rs}_2, \textit{tq}.(\textit{os}_1, \textit{pc}_1, \textit{e}_1, \textit{rs}_1), c)}\\\)
If a thread times out and has not finished execution (has not executed the \texttt{RET} statement), then it is enqueued on the thread queue, and the next thread is dequeued from the thread queue, with a constant timeout value \(c\).

The above rule assumes there is least one thread in the thread queue. To cover all cases, here is the rule for zero threads in the thread queue:
\(\\\Rule{}{(\textit{os}, \textit{pc}, \textit{e}, \textit{rs}, \langle \rangle, 0) \transition (\textit{os}, \textit{pc}, \textit{e}, \textit{rs}, \langle \rangle, c)}\\\)

\paragraph{Running thread}
\label{sec:orgfa4cc96}
\(\\\Rule{s(\textit{pc}) \neq \texttt{RET} \qquad \textit{to} > 0}{(\textit{os}, \textit{pc}, \textit{e}, \textit{rs}, \textit{tq}, \textit{to}) \transition (\textit{os'}, \textit{pc'}, \textit{e'}, \textit{rs'}, \textit{tq}, \textit{to}-1)}\\\)
where the primed values are just like normal VM code execution, and the timeout is initially nonzero, and then decrements.

\paragraph{Running thread, returning from function}
\label{sec:orgc0cd6b5}
\(\\\Rule{s(\textit{pc}) = \texttt{RET} \qquad \textit{to} > 0 \qquad \textit{rs} \neq \langle \rangle}{(\textit{os}, \textit{pc}, \textit{e}, \textit{rs}, \textit{tq}, \textit{to}) \transition (\textit{os'}, \textit{pc'}, \textit{e'}, \textit{rs'}, \textit{tq}, \textit{to}-1)}\\\)
where the primed values are just like normal VM code execution, and the timeout is initially nonzero, and then decrements. Note: the thread may execute the \texttt{RET} statement inside a function, and the thread does the normal thing of popping \texttt{rs} and so on.

\paragraph{Starting \texttt{EXECUTE}}
\label{sec:org6246ae4}
\(\\\Rule{s(\textit{pc}) = \texttt{EXECUTE n} \qquad \textit{to} > 0}{((\textit{pc}_1, \textit{e}_1). \cdots .(\textit{pc}_n, \textit{e}_n).\textit{os}, \textit{pc}, \textit{e}, \textit{rs}, \langle \rangle, \textit{to})}\\\)
\(\transition (\textit{os}_j, \textit{pc}_j, \textit{e}_j, \textit{rs}_j, (\langle \rangle, \textit{pc}_1, \textit{e}_1, \langle \rangle). \cdots .(\langle \rangle, \textit{pc}_n, \textit{e}_n, \langle \rangle), c)\\\)
Closures representing threads \(i\) (two-tuples of \(\textit{pc}_i\) and \(\textit{e}_i\)) on the operand stack are converted into threads \(i\). Thread \(i\) is a four-tuple of each thread \(i\)'s own \(\textit{os}_i\), \(\textit{pc}_i\), \(\textit{e}_i\), and \(\textit{rs}_i\). Initially, thread \(i\) has empty \(\textit{os}_i\) and empty \(\textit{rs}_i\).
The thread that calls \texttt{concurrent\_execute} also continues to execute concurrently with the other threads. This is shown by the disappearance of \(\textit{os}, \textit{pc}, \textit{e}, \textit{rs}\), meaning that the thread that calls \texttt{concurrent\_execute} is enqueued on the thread queue.
Some next thread is dequeued from the thread queue, \(\textit{os}_j, \textit{pc}_j, \textit{e}_j, \textit{rs}_j\), with a constant timeout value \(c\).

\paragraph{Returning from thread}
\label{sec:orgb558241}
\(\\\Rule{s(\textit{pc}_1) = \texttt{RET} \qquad \textit{to} > 0 \qquad \textit{rs}_1 = \langle \rangle}{(\textit{os}_1, \textit{pc}_1, \textit{e}_1, \textit{rs}_1, (\textit{os}_2, \textit{pc}_2, \textit{e}_2, \textit{rs}_2).\textit{tq}, 0) \transition (\textit{os}_2, \textit{pc}_2, \textit{e}_2, \textit{rs}_2, \textit{tq}, c)}\\\)
If a thread executes the \texttt{RET} statement, and the runtime stack is empty, then the thread is not enqueued on the thread queue, and the next thread is dequeued from the thread queue, with a constant timeout value \(c\).

The above rule assumes there is least one thread in the thread queue. To cover all cases, the rule for zero threads in the thread queue is in the next subsection:

\paragraph{Ending our VM}
\label{sec:org2dae0a8}
\(\\\Rule{s(\textit{pc}) = \texttt{RET} \qquad \textit{to} > 0 \qquad \textit{rs} = \langle \rangle \qquad \textit{tq} = \langle \rangle}{(\textit{os}, \textit{pc}, \textit{e}, \textit{rs}, \textit{tq}, \textit{to}) \transition (\textit{os}, \textit{pc}, \textit{e}, \textit{rs}, \textit{tq}, \textit{to}-1)}\\\)
If a thread executes the \texttt{RET} statement, and both the runtime stack and the thread queue are empty, and the timeout is nonzero, then the timeout decrements, and our VM stops.

\subsection{\texttt{TEST\_AND\_SET} and \texttt{CLEAR} Rules}
\label{sec:org3306985}

\subsubsection{Notes}
\label{sec:orgcccda22}
\begin{itemize}
\item For simplicity, all registers and heap are not represented in the rules, except \(\textit{os}\) and \(\textit{pc}\).
\item \texttt{test\_and\_set} is an atomic operation.
\end{itemize}

\subsubsection{Compiling}
\label{sec:org025d9fd}
\(\Rule{E \translateaux s}{\texttt{test\_and\_set}(E) \translateaux s.\texttt{TEST\_AND\_SET}}\\\)
where \(E\) is an array, whose head is a boolean.

\(\Rule{E \translateaux s}{\texttt{clear}(E) \translateaux s.\texttt{CLEAR}}\\\)
where \(E\) is an array.

\subsubsection{Running}
\label{sec:org6d01845}
\(\Rule{s(\textit{pc}) = \texttt{TEST\_AND\_SET}}{(a.\textit{os},\textit{pc}) \transition (b.\textit{os},\textit{pc} + 1)}\\\)
where \(a\) is the address of an array stored on the heap. The head of this array is initially \(b\), where \(b\) is a boolean. After this rule executes, the head of this array is set to \(\textit{true}\).

\(\Rule{s(\textit{pc}) = \texttt{CLEAR}}{(a.\textit{os},\textit{pc}) \transition (\textit{os},\textit{pc} + 1)}\\\)
where \(a\) is the address of an array stored on the heap. The head of this array is updated to \(\textit{false}\).
\end{document}
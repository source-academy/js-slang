\input source_header.tex

\begin{document}
	%%%%%%%%%%%%%%%%%%%%%%%%%%%%%%%%%%%%%%%%%%%%%%%
	\docheader{2021}{Source}{\S 4 GPU}{Martin Henz, Rahul Rajesh, Zhang Yuntong, Lim Ao Jun Joel, Lim Haw Jia}
	%%%%%%%%%%%%%%%%%%%%%%%%%%%%%%%%%%%%%%%%%%%%%%%

\input source_intro.tex

\section{Changes}

Source \S 4 GPU allows for Source programs to be accelerated on the GPU if certain conditions are met.
The exact specifications for this is outlined on page \pageref{gpu_supp}. Source \S 4 GPU  defines a formal specification 
to identify areas in the program that are embarrassingly parallel (e.g. for loops etc.) . These will then
be run in parallel across GPU threads. Experimentation has shown that Source \S 4 GPU is orders of magnitude faster
than Source \S 4 for heavy CPU bound tasks (matrix multiplication of large matrices)

\input source_bnf.tex

\input source_3_bnf.tex

\newpage

\input source_return

\input source_import

\input source_boolean_operators

\input source_loops

\input source_names_lang

\input source_numbers

\input source_strings

\input source_arrays

\input source_comments

\input source_typing_3

\section{Standard Libraries}

The following libraries are always available in this language.

\input source_misc

\input source_math

\input source_lists

\input source_pair_mutators

\input source_array_support

\input source_streams

\input source_interpreter

\input source_js_differences

\newpage

\section*{GPU Acceleration}
\label{gpu_supp}
This section outlines the specifications for programs to be accelerated using the GPU.\
\input source_gpu_bnf.tex

\newpage

\section*{Restrictions}

Even if the BNF syntax is met, GPU acceleration can only take place if all the restrictions below are satisfied. If all criteria are met, the \textit{gpu\_statement} loops are embarrassingly parallel.

\subsection*{Special For Loops}

In the BNF, we have special loops that take on this form:
\begin{alignat*}{9}
&& \textbf{\texttt{for}}\ \textbf{\texttt{(}} 
                          \ \textit{gpu\_for\_let} \textbf{\texttt{;}} \\
&& \ \ \textit{gpu\_condition} \textbf{\texttt{;}} \\
&& \textit{gpu\_for\_assignment} \ \textbf{\texttt{)}} 
\end{alignat*}

These are the loops that will be taken into consideration for parallelization. However, on top of the BNF syntax, the below requirement must also be satisfied:

\begin{itemize}
    \item{the names declared in each \textit{gpu\_for\_let} have to be different across the loops}
    \item{in each loop, the \textit{gpu\_condition} and the \textit{gpu\_for\_assignment} must use the name declared
    in the respective \textit{gpu\_for\_let} statement}
    \item{the initial value of the counter in  \textit{gpu\_for\_let} and the value of the terminal value in \textit{gpu\_condition} must be non-negative}
    \item{the counter must increment or decrement in the direction of the terminal value}
    \item{the initial value of the counter in \textit{gpu\_for\_let} and the step size in \textit{gpu\_for\_assignment} must be integers}
\end{itemize}

\subsection*{GPU Function}

A \textit{gpu\_function} can be a \textit{math\_\texttt{*}} function or a user-defined function.
If it is a user-defined function, it must satisfy all of these constraints:

\begin{itemize}
    \item{the function name must not be one of the reserved keywords in the OpenGL Shading Language\footnote{See \href{ https://www.khronos.org/registry/OpenGL/specs/es/2.0/GLSL_ES_Specification_1.00.pdf}{OpenGL ES Shading Language specifications}, Section 3.7.}}
    \item{there must not be any use of external variables defined outside of the function scope}
    \item{the function must not be a higher-order function (it may not take any functions as arguments)}
    \item{the function must not be recursive. This also includes the case of indirect recursion, so the function may not call itself, and it may not call a chain of functions that eventually calls itself (e.g. A calls B, B calls C, C calls A)}
    \item{the function must not use any of Source's standard libraries (other than the math library)}
    \item{the function must be declared in the global scope of the program}
    \item{any function that is called within this function must also be a valid \textit{gpu\_function}}
\end{itemize}


\subsection*{Core Statement}

Within \textit{core\_statement}, there are some constraints:

\begin{itemize}
    \item{no assignment to any global variables (all assignments can only be done to variables defined in the \textit{gpu\_block}})
    \item{no use of the variable in \textit{gpu\_result\_assignment} at an offset from the current index e.g. cannot be i - 1}
\end{itemize}

\subsection*{GPU Result Statement}

The \textit{gpu\_result\_assignment} is the statement that stores a value calculated in core statements into a result array. 
It access an array at a certain coordinate e.g. ${array[{i_1}][{i_2}][{i_3}]}$. For this: 

\begin{itemize}
    \item{This result array has to be defined outside the \textit{gpu\_block}.}
    \item{ If you have ${n}$ special for loops, the array expression can take on ${k}$ coordinates where ${0 < k \leq n}$.} 
    \item{ The coordinate values can be either a number, an external variable or a counter only. }
\end{itemize}

\section*{Examples}

Below are some examples of valid and invalid source gpu programs:\\

\textbf{Valid} - Using first loop counter. (meaning the loop will be run across N threads; the first loop is parallelized away):
\begin{verbatim}
for (let i = 0; i < N; i = i + 1) {
    for (let k = 0; k < M; k = k + 1) {
        res[i] = arr[k % 2] + 1;
    }
}
\end{verbatim}

\textbf{Valid} - Using first three loop counters (meaning the loop will be run across N*M*C threads, if available):
\begin{verbatim}
for (let i = 0; i < N; i = i + 1) {
    for (let j = 0; j < M; j = j + 1) {
        for (let k = 0; k < C; k = k + 1) {
            let x = math_pow(2, 10);
            let y = x * (1000);
            arr[i][k][j] = (x + y * 2);
        }
    }
}
\end{verbatim}

\textbf{Valid} - Using valid user-defined functions within the loop body:
\begin{verbatim}
function add(x, y) {
    return x + y;
}
function multiply(x, y) {
    return x * y;
}
for (let i = 0; i < N; i = i + 1) {
    for (let j = 0; j < M; j = j + 1) {
        for (let k = 0; k < C; k = k + 1) {
             res[i][j][k] = multiply(add(i, j), k);
        }
    }
}
\end{verbatim}

\textbf{Invalid} - Using recursive functions within the loop body:
\begin{verbatim}
function f(x) {
    if (x === 0) {
        return 0;
    } else {
        return g(x - 1);
    }
}
function g(x) {
    if (x === 0) {
        return 0;
    } else {
        return f(x - 1);
    }
}

for (let i = 0; i < N; i = i + 1) {
    res[i] = f(i);
}
\end{verbatim}

\textbf{Invalid} - Using functions which access external variables:
\begin{verbatim}
let y = 1;
function f(x) {
    return x + y;
}

for (let i = 0; i < N; i = i + 1) {
    res[i] = f(i);
}
\end{verbatim}

\textbf{Valid} - Using a for loop where the counter starts as, and increments/decrements by an arbitrary integer value:
\begin{verbatim}
for (let i = 0; i < N; i = i + 1) {
    for (let j = M; j >= 2; j = j - 3) {
        for (let k = 1; k < C; k = k + 2) {
            res[i][j][k] = i * j * k;
        }
    }
}
\end{verbatim}

\textbf{Invalid} - Using a for loop where the terminal value is negative:
\begin{verbatim}
for (let i = 0; i < N; i = i + 1) {
    for (let j = 0; j < M; j = j + 1) {
        for (let k = C; k >= -1 ; k = k - 2) {
            res[i][j][k] = i * j * k;
        }
    }
}
\end{verbatim}

\newpage

\input source_list_library

\newpage

\input source_stream_library


\end{document}

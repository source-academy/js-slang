\subsection*{Names}

Names start with \verb@_@ or a
letter\footnote{
By \emph{letter}
we mean \href{http://unicode.org/reports/tr44/}{\color{DarkBlue}Unicode} letters (L) or letter numbers (NI).
} and contain only \verb@_@
or alphanumeric characters\footnote{
By alphanumeric characters we mean A-z, 0-9
}. Restricted words\footnote{ By \emph{restricted word} we mean any of:$
\textbf{\texttt{False}}$, $\textbf{\texttt{None}}$, $\textbf{\texttt{True}}$, $\textbf{\texttt{and}}$, $\textbf{\texttt{as}}$, $\textbf{\texttt{assert}}$, $\textbf{\texttt{async}}$, $\textbf{\texttt{await}}$, $\textbf{\texttt{break}}$, $\textbf{\texttt{class}}$, $\textbf{\texttt{continue}}$, $\textbf{\texttt{def}}$, $\textbf{\texttt{del}}$, $\textbf{\texttt{elif}}$, $\textbf{\texttt{else}}$, $\textbf{\texttt{except}}$, $\textbf{\texttt{finally}}$, $\textbf{\texttt{for}}$, $\textbf{\texttt{from}}$, $\textbf{\texttt{global}}$, $\textbf{\texttt{if}}$, $\textbf{\texttt{import}}$, $\textbf{\texttt{in}}$, $\textbf{\texttt{is}}$, $\textbf{\texttt{lambda}}$, $\textbf{\texttt{nonlocal}}$, $\textbf{\texttt{not}}$, $\textbf{\texttt{or}}$, $\textbf{\texttt{pass}}$, $\textbf{\texttt{raise}}$, $\textbf{\texttt{return}}$, $\textbf{\texttt{try}}$, $\textbf{\texttt{while}}$, $\textbf{\texttt{with}}$, $\textbf{\texttt{yield}}$.
These are all words that cannot be used without restrictions as names in the strict mode of ECMAScript 2020.
} are not allowed as names.

Valid names are \verb@x@, \verb@_x@, \verb@X@ and \verb@X_@,
but always keep in mind that programming is communicating and that the familiarity of the
audience with the characters used in names is an important aspect of program readability.


\section{Dynamic Type Checking}

Expressions evaluate to numbers, boolean values, strings or function values. Implementations
of Python \S 1 generate error messages when unexpected values are used as follows.

Only function values can be applied using the syntax:

\begin{eqnarray*}
 \textit{expression}    
                                   & ::=   &  \textit{name}
                                               \texttt{\textbf{(}}\  \textit{expressions} \
                                               \texttt{\textbf{)}}\\ 
\end{eqnarray*}

For compound functions, implementations need to check that the number of \textit{expressions}
matches the number of parameters.

The following table specifies what arguments  Python \S 1's operators
take and what results they return. Implementations need to check the types of arguments and
generate an error message when the types do not match.

\begin{center}
\begin{tabular}{c|c|c|c}
\textbf{operator} & \textbf{argument 1} & \textbf{argument 2} & \textbf{result} \\ \hline 
\texttt{+} & int & int & int \\
\texttt{+} & float & int | float & float \\
\texttt{+} & string & string & string \\
\texttt{-} & int & int & int \\
\texttt{-} & float & int | float & float \\
\texttt{*} & int & int & int \\
\texttt{*} & float & int | float & float \\
\texttt{*} & string & int & string \\
\texttt{/} & int | float & int | float & float \\
\texttt{\%} & int & int & int \\
\texttt{\%} & float & int | float & float \\
\texttt{**} & int & int & int \\
\texttt{**} & float & int | float & float \\
\texttt{//} & int | float & int | float & int \\
\texttt{==} & any & any & bool \\
\texttt{!=} & any & any & bool \\
\texttt{>} & int | float & int | float & bool \\
\texttt{>} & string & string & bool \\
\texttt{<} & int | float & int | float & bool \\
\texttt{<} & string & string & bool \\
\texttt{>=} & int | float & int | float & bool \\
\texttt{>=} & string & string & bool \\
\texttt{<=} & int | float & int | float & bool \\
\texttt{<=} & string & string & bool \\
\texttt{and} & bool & bool & bool \\
\texttt{or} & bool & bool & bool \\
\texttt{not} & bool & N/A & bool \\
\texttt{-} (unary) & int & N/A & int \\
\texttt{-} (unary) & float & N/A & float \\
\end{tabular}
\end{center}

Note: Python supports more complex behaviors through operator overloading and special methods (e.g., \_\_add\_\_, \_\_lt\_\_, etc.), allowing custom objects to interact with these operators in user-defined ways. This table covers only the most basic and common use cases.
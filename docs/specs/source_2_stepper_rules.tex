\input source_2_stepper_style.tex

\section*{Reduction}

The \emph{reducer} $\Rightarrow$ is a partial function from
programs/statements/expressions to
programs/statements/expressions (with slight abuse of notation
via overloading)
and $\Rightarrow^*$ is its reflexive transitive closure.
A \emph{reduction} is a sequence of programs
$p_1 \Rightarrow \cdots \Rightarrow p_n$,
where $p_n$ is not reducible, i.e. there is no
program $q$ such that $p_n \Rightarrow q$.
Here, the program $p_n$ is called the \emph{result
of reducing} $p_1$.

If $p_n$, the result of reducing a program, is of the form \sCode{\sVar{v};}, 
we call \sVar{v} the result of the program evaluation. If $p_n$ is the empty
sequence of statements, we declare the result of the program evaluation to be
the value undefined. Reduction can get "stuck": the result can be a program
that has neither of these two forms, which corresponds to a runtime error.

A \emph{value} is a primitive number expression,
primitive boolean expression,
a primitive string expression, a function definition
expression or a function declaration statement.

The \emph{substitution} function
$p [ n := v ]$ on programs/statements/expressions
replaces every free occurrence of the name $n$
in statement $p$ by value $v$. Care must be taken to introduce
and preserve
co-references in this process; substitution can introduce
cyclic references in the result of the substitution. For example,
$n$ may occur free in $v$, in which case
every occurrence of $n$ in $p$
will be replaced by $v$ such that $n$ in $v$ refers cyclically
to the node at which the replacement happens.

\subsection*{Programs}

\sEntry{Program-intro}{In a sequence of statements, we can always reduce the
    first one that isn't a value statement.}
\gentzen{
    \sVar{statement} \Rightarrow \sVar{statement}'
}{
    [\sVar{value}\sKw{;}]\sKw{ }\sVar{statement}\ldots \Rightarrow
    [\sVar{value}\sKw{;}]\sKw{ }\sVar{statement}'\ldots
}

\sEntry{Program-reduce}{When the first two statements in a program are value
    statements, the first one can be removed.}
\gentzen{
    \sVar{v1}, \sVar{v2} \mbox{ are values}
}{
    \sVar{v1}\sKw{; }\sVar{v2}\sKw{; }\sVar{statement}\ldots \Rightarrow
    \sVar{v2}\sKw{; }\sVar{statement}\ldots
}

\vspace{10mm}

% Weird corner case: Doesn't affect output, but if the leading value is a
% lambda, we prob should substitute the fn/constant in that lambda as well?
\textbf{Eliminate-function-declaration}: Function declarations as the first
    statement optionally after one value statement are substituted in the
    remaining statements.
\[
\frac{
             f = \textbf{\texttt{function}}\  \textit{name} \ 
                 \textbf{\texttt{(}}\  \textit{parameters}
                 \ \textbf{\texttt{)}}\ \textit{block}
}{
  [ \sVar{value} \sKw{;} ] \sKw{ }
  f\ \textit{statement} \ldots\ 
  \Rightarrow\ 
  [ \sVar{value} \sKw{;} ] \sKw{ }
  \textit{statement} \ldots[\textit{name} := f]
}
\]

% See weird corner case above
\sEntry{Eliminate-constant-declaration}{Constant declarations as the first
    statement optionally after one value statement are substituted in the
    remaining statements.}
\[
\frac{
             c = \textbf{\texttt{const}}\  \textit{name} \ 
             \textbf{\texttt{=}}\  \textit{v}
}{
  [ \sVar{value} \sKw{;} ] \sKw{ }
  c\ \textit{statement} \ldots\ 
  \Rightarrow \ 
  [ \sVar{value} \sKw{;} ] \sKw{ }
  \textit{statement} \ldots[\textit{name} := v]
}
\]

\subsection*{Statements: Expression statements}

\textbf{Expression-statement-reduce}: An expression statement
is reducible if its expression is reducible.
\[
\frac{
  e\ \Rightarrow\ e'
}{  
  e \textbf{\texttt{;}}
  \ \Rightarrow \ 
  e' \textbf{\texttt{;}}
}
\]

\subsection*{Statements: Constant declarations}

\textbf{Evaluate-constant-declaration}: The right-hand expressions
in constant declarations are evaluated.
\[
\frac{
  \textit{expression}
  \ \Rightarrow \ 
  \textit{expression}'
}{
  \textbf{\texttt{const}}\  \textit{name} \ 
  \textbf{\texttt{=}}\  \textit{expression} \ 
  \Rightarrow \ 
  \textbf{\texttt{const}}\  \textit{name} \ 
  \textbf{\texttt{=}}\  \textit{expression}'
}
\]


\subsection*{Statements: Conditionals}

\textbf{Conditional-statement-predicate}: A conditional statement
is reducible if its predicate is reducible.
\[
\frac{
  e\ \Rightarrow\ e'
}{  
  \textbf{\texttt{if}}\ 
  \textbf{\texttt{(}}\ 
  e \ 
  \textbf{\texttt{)}} \ 
  \textbf{\texttt{\{}} \
  \cdots
  \textbf{\texttt{\}}} \ 
  \textbf{\texttt{else}} \ 
  \textbf{\texttt{\{}} \
  \cdots
  \textbf{\texttt{\}}}
\ 
  \Rightarrow \ 
  \textbf{\texttt{if}}\ 
  \textbf{\texttt{(}}\ 
  e' \ 
  \textbf{\texttt{)}} \ 
  \textbf{\texttt{\{}} \
  \cdots
  \textbf{\texttt{\}}} \ 
  \textbf{\texttt{else}} \ 
  \textbf{\texttt{\{}} \
  \cdots
  \textbf{\texttt{\}}}
}
\]

\vspace{10mm}

SICP-JS Exercise 4.8 specifies that a conditional statement where the taken
branch is non-value-producing is value-producing, with its value being
\sKw{undefined}.

\sEntry{Conditional-statement-consequent}{Immediately within a program or block
    statement, a conditional statement whose predicate is true reduces to the
    consequent block.}
\[
\frac{
}{  
  \textbf{\texttt{if}}\ 
  \textbf{\texttt{(}}\ 
  \textbf{\texttt{true}} \ 
  \textbf{\texttt{)}} \ 
  \textbf{\texttt{\{}} \
  \textit{program}_1 
  \textbf{\texttt{\}}} \ 
  \textbf{\texttt{else}} \ 
  \textbf{\texttt{\{}} \
\textit{program}_2 
  \textbf{\texttt{\}}}
  \ 
  \Rightarrow  \ 
  \textbf{\texttt{\{}} \
  \textbf{\texttt{undefined; }} \
  \textit{program}_1 
  \textbf{\texttt{\}}}
}
\]

\sEntry{Conditional-statement-alternative}{Immediately within a program or block
    statement, a conditional statement whose predicate is false reduces to the
    alternative block.}
\[
\frac{
}{  
  \textbf{\texttt{if}}\ 
  \textbf{\texttt{(}}\ 
  \textbf{\texttt{false}} \ 
  \textbf{\texttt{)}} \ 
  \textbf{\texttt{\{}} \
  \textit{program}_1
  \textbf{\texttt{\}}} \ 
  \textbf{\texttt{else}} \ 
  \textbf{\texttt{\{}} \
\textit{program}_2  
  \textbf{\texttt{\}}}
\ \Rightarrow \ 
  \textbf{\texttt{\{}} \
  \textbf{\texttt{undefined; }} \
  \textit{program}_2 
  \textbf{\texttt{\}}}
}
\]

We can remove the \sCode{undefined;} if the conditional statement is in a block
expression without affecting the return value of the block expression. This is
performed to reduce the surprise factor.

\sEntry{Conditional-statement-blockexpr-consequent}{Immediately within a block
    expression, a conditional statement whose predicate is true reduces to the
    consequent block.}
\[
\frac{
}{  
  \textbf{\texttt{\{}}\ 
  [
  \textit{value}
  \textbf{\texttt{;}}
  ]\ 
  \textbf{\texttt{if}}\ 
  \textbf{\texttt{(}}\ 
  \textbf{\texttt{true}} \ 
  \textbf{\texttt{)}} \ 
  \textbf{\texttt{\{}} \
  \textit{program}_1 
  \textbf{\texttt{\}}} \ 
  \textbf{\texttt{else}} \ 
  \textbf{\texttt{\{}} \ 
  \textit{program}_2
  \textbf{\texttt{\}}}\ 
  \textbf{\texttt{\}}}
  \ 
  \Rightarrow  \ 
  \textbf{\texttt{\{}}
  [
  \textit{value}
  \textbf{\texttt{;}}
  ]\ 
  \textbf{\texttt{\{}}\ 
  \textit{program}_1
  \textbf{\texttt{\}}}
  \textbf{\texttt{\}}}
}
\]

\sEntry{Conditional-statement-blockexpr-alternative}{Immediately within a block
    expression, a conditional statement whose predicate is false reduces to the
    alternative block.}
\[
\frac{
}{  
  \textbf{\texttt{\{}}\ 
  [
  \textit{value}
  \textbf{\texttt{;}}
  ]\ 
  \textbf{\texttt{if}}\ 
  \textbf{\texttt{(}}\ 
  \textbf{\texttt{false}} \ 
  \textbf{\texttt{)}} \ 
  \textbf{\texttt{\{}} \
  \textit{program}_1
  \textbf{\texttt{\}}} \ 
  \textbf{\texttt{else}} \ 
  \textbf{\texttt{\{}} \
  \textit{program}_2
  \textbf{\texttt{\}}}\ 
  \textbf{\texttt{\}}}
  \ \Rightarrow \ 
  \textbf{\texttt{\{}}
  [
  \textit{value}
  \textbf{\texttt{;}}
  ]\ 
  \textbf{\texttt{\{}}\ 
  \textit{program}_2
  \textbf{\texttt{\}}}
  \textbf{\texttt{\}}}
}
\]

\subsection*{Statements: Blocks}

\textbf{Block-statement-intro}: A block statement is
reducible if its body is reducible.
\[
\frac{
  \sVar{program}
\ \Rightarrow \ 
  \sVar{program}'
}{  
  \textbf{\texttt{\{}} \
  \sVar{program} \ 
  \textbf{\texttt{\}}}
\  \Rightarrow \ 
  \textbf{\texttt{\{}} \
  \sVar{program}' \ 
  \textbf{\texttt{\}}}
}
\]

\vspace{10mm}

\sEntry{Block-statement-single-reduce}{A block statement consisting of a single
    value statement reduces to that value statement.}
\gentzen{}{
    \sKw{\{}\sVar{value}\sKw{;\}} \Rightarrow \sVar{value}\sKw{;}
}

\sEntry{Block-statement-empty-reduce}{An empty block statement (or a non-value-
    producing one, which will reduce to the empty block statement) is
    non-value-producing, i.e. reduces to $\varepsilon$.}
\gentzen{}{
    \sKw{\{} \sKw{\}} \Rightarrow \varepsilon
}

\subsection*{Expresssions: Blocks}

\textbf{Block-expression-intro}: A block expression is
reducible if its body is reducible.
\[
\frac{
  \sVar{program} 
\ \Rightarrow \ 
  \sVar{program}'
}{  
  \textbf{\texttt{\{}} \
  \sVar{program} \ 
  \textbf{\texttt{\}}}
\  \Rightarrow \ 
  \textbf{\texttt{\{}} \
  \sVar{program}' \ 
  \textbf{\texttt{\}}}
}
\]

\vspace{10mm}

\sEntry{Block-expression-single-reduce}{A block expression consisting of a
    single value statement reduces to \sKw{undefined}.}
\gentzen{}{
    \sCode{\{\sVar{value};\}} \Rightarrow \sCode{undefined}
}

\sEntry{Block-expression-empty-reduce}{An empty block expression reduces to
    \sKw{undefined}.}
\gentzen{}{
    \sCode{\{\}} \Rightarrow \sCode{undefined}
}

\sEntry{Block-expression-return-reduce-1}{In a block expression starting with
    a value statement and then a return statement, the value statement can be
    removed.}

\gentzen{}{
    \sCode{
        \{\sVar{value};
        return \sVar{return-expression}; 
        \sVar{statement}$\ldots$\}
    }
    \Rightarrow
    \sCode{
        \{return \sVar{return-expression}; \sVar{statement}$\ldots$\}
    }
}

\sEntry{Block-expression-return-reduce-2}{A return statement as the first
    statement in a block expression can be reduced by reducing the return
    expression.}
\gentzen{
  \sVar{return-expression}
\ \Rightarrow \ 
  \sVar{return-expression}'
}{
  \sKw{\{} \sKw{return } \sVar{return-expression} \sKw{; }
  \sVar{statement\ldots} \sKw{\}}
  \Rightarrow
  \sKw{\{} \sKw{return } \sVar{return-expression}' \sKw{; }
  \sVar{statement\ldots} \sKw{\}}
}

\sEntry{Block-expression-return-reduce-3}{A block expression starting with a
    return statement reduces to the value of the return statement.}
\gentzen{}{
    \sKw{\{} \sKw{return } \sVar{value} \sKw{; }
    \sVar{statement\ldots} \sKw{\}}
    \Rightarrow
    \sVar{value}
}

Block expressions are currently used only as expanded forms of functions.

\subsection*{Expressions: Binary operators}

\textbf{Left-binary-reduce}: An expression with binary operator
can be reduced if its left sub-expression can be reduced.
\[
\frac{
  e_1 \ \Rightarrow \ e_1'
}{
  e_1\  \textit{binary-operator} \ e_2
  \ \Rightarrow \ 
  e_1'\  \textit{binary-operator} \ e_2
}
\]


\vspace{10mm}
\textbf{And-shortcut-false}: An expression with binary operator
$\textbf{\texttt{\&\&}}$ whose left sub-expression is
$\textbf{\texttt{false}}$ can be reduced to
$\textbf{\texttt{false}}$.
\[
\frac{
}{
  \textbf{\texttt{false}}\  \textbf{\texttt{\&\&}}\ e
  \ \Rightarrow \ 
  \textbf{\texttt{false}}
}
\]

\vspace{10mm}
\textbf{And-shortcut-true}: An expression with binary operator
$\textbf{\texttt{\&\&}}$ whose left sub-expression is
$\textbf{\texttt{true}}$ can be reduced to
the right sub-expression.
\[
\frac{
}{
  \textbf{\texttt{true}}\  \textbf{\texttt{\&\&}}\ e
  \ \Rightarrow \ 
  e
}
\]

\vspace{10mm}
\textbf{Or-shortcut-true}: An expression with binary operator
$\textbf{\texttt{||}}$ whose left sub-expression is
$\textbf{\texttt{true}}$ can be reduced to
$\textbf{\texttt{true}}$.
\[
\frac{
}{
  \textbf{\texttt{true}}\  \textbf{\texttt{||}}\ e
  \ \Rightarrow \ 
  \textbf{\texttt{true}}
}
\]

\vspace{10mm}
\textbf{Or-shortcut-false}: An expression with binary operator
$\textbf{\texttt{||}}$ whose left sub-expression is
$\textbf{\texttt{false}}$ can be reduced to
the right sub-expression.
\[
\frac{
}{
  \textbf{\texttt{false}}\  \textbf{\texttt{||}}\ e
  \ \Rightarrow \ 
  e
}
\]

\vspace{10mm}
\textbf{Right-binary-reduce}: An expression with binary operator
can be reduced if its left sub-expression is a value and its right
sub-expression can be reduced.
\[
\frac{
  e_2\ \Rightarrow\ e_2', \textrm{and}\ \textit{binary-operator}
  \mbox{is not}\ \textbf{\texttt{\&\&}}\ \textrm{or}\ \texttt{\textbf{||}}
}{
  v\  \textit{binary-operator} \ e_2
  \ \Rightarrow \ 
  v\  \textit{binary-operator} \ e_2'
}
\]

\vspace{10mm}
\textbf{Prim-binary-reduce}: An expression with binary operator
can be reduced if its left and right sub-expressions are values and
the corresponding function is defined for those values.
\[
\frac{
  v\ \mbox{is result of}\ v_1\  \textit{binary-operator} \ v_2
}{
  v_1\  \textit{binary-operator} \ v_2
  \ \Rightarrow \ 
  v
}
\]

\subsection*{Expressions: Unary operators}

\textbf{Unary-reduce}: An expression with unary operator
can be reduced if its sub-expression can be reduced.
\[
\frac{
  e \ \Rightarrow \ e'
}{
  \textit{unary-operator} \ e
  \ \Rightarrow \ 
  \textit{unary-operator} \ e'
}
\]

\vspace{10mm}
\textbf{Prim-unary-reduce}: An expression with unary operator
can be reduced if its sub-expression is a value and
the corresponding function is defined for that value.
\[
\frac{
  v'\ \mbox{is result of}\ \textit{unary-operator} \ v
}{
  \textit{unary-operator} \ v
  \ \Rightarrow \ 
  v'
}
\]

\subsection*{Expressions: conditionals}

\textbf{Conditional-predicate-reduce}: A conditional
expression can be reduced, if its predicate can be reduced.
\[
\frac{
  e_1 \ \Rightarrow \ e_1'
}{
  e_1\  \textbf{\texttt{?}}\ e_2\ \textbf{\texttt{:}}\ e_3
  \ \Rightarrow \ 
  e_1'\ \textbf{\texttt{?}}\ e_2\ \textbf{\texttt{:}}\ e_3
}
\]

\vspace{10mm}
\textbf{Conditional-true-reduce}: A conditional
expression whose predicate is the value
$\textbf{\texttt{true}}$
can be reduced to its consequent expression.
\[
\frac{
}{
  \textbf{\texttt{true}}\  \textbf{\texttt{?}}\ e_1\ \textbf{\texttt{:}}\ e_2
  \ \Rightarrow \ 
  e_1
}
\]

\vspace{10mm}
\textbf{Conditional-false-reduce}: A conditional
expression whose predicate is the value
$\textbf{\texttt{false}}$
can be reduced to its alternative expression.
\[
\frac{
}{
  \textbf{\texttt{false}}\  \textbf{\texttt{?}}\ e_1\ \textbf{\texttt{:}}\ e_2
  \ \Rightarrow \ 
  e_2
}
\]


\subsection*{Expressions: function application}

\textbf{Application-functor-reduce}: A function application
can be reduced if its functor expression can be reduced.
\[
\frac{
  e \ \Rightarrow \ e'
}{
  e\  \textbf{\texttt{(}}\ \textit{expressions} \ \textbf{\texttt{)}}
  \ \Rightarrow \ 
  e'\  \textbf{\texttt{(}}\ \textit{expressions} \ \textbf{\texttt{)}}
}
\]


\vspace{10mm}
\textbf{Application-argument-reduce}: A function application
can be reduced if one of its argument expressions can be reduced and all
preceding arguments are values.
\[
\frac{
  e \ \Rightarrow \ e'
}{
  v\  \textbf{\texttt{(}}\ v_1 \ldots v_i \ e \ldots\ \textbf{\texttt{)}}
  \ \Rightarrow \ 
  v\  \textbf{\texttt{(}}\ v_1 \ldots v_i \ e' \ldots\ \textbf{\texttt{)}}
}
\]



\vspace{10mm}
\textbf{Function-declaration-application-reduce}:
The application of a function declaration
can be reduced, if all
arguments are values. 
\[
\frac{
  f = \textbf{\texttt{function}}\  \textit{n} \ 
                 \textbf{\texttt{(}}\  x_1 \ldots x_n
                 \ \textbf{\texttt{)}}\ \textit{block}
}{
  f\ \textbf{\texttt{(}}\ v_1 \ldots v_n\ \textbf{\texttt{)}}
  \ \Rightarrow \ 
  \textit{block} [x_1 := v_1]\ldots[x_n := v_n]
  [n := f]
}
\]

\vspace{10mm}
\textbf{Function-definition-application-reduce}:
The application of a function definition
can be reduced, if all
arguments are values. 
\[
\frac{
  f = \textbf{\texttt{(}}\  x_1 \ldots x_n
                 \ \textbf{\texttt{)}}\ \textbf{\texttt{=>}}\ b
                   \mbox{, where $b$ is an expression or block}
}{
  f\ \textbf{\texttt{(}}\ v_1 \ldots v_n\ \textbf{\texttt{)}}
  \ \Rightarrow \ 
  b [x_1 := v_1]\ldots[x_n := v_n]
}
\]

\pagebreak

\section*{Substitution}

\textbf{Identifier}: An identifier with the same name as $x$ is substituted with $e_x$.
\[
\frac{
}{
  x[x := e_x]
  \ = \ 
  e_x
}
\]

\[
\frac{
\textit{name}
\ \neq \ 
x
}{
  \textit{name} [x := e_x] 
  \ = \ 
  \textit{name}
}
\]

\vspace{10mm}
\textbf{Expression statement}: All occurrences of $x$ in $e$ are substituted with $e_x$.
\[
\frac{
}{
  e\textbf{\texttt{;}}[x := e_x]
  \ = \ 
  e\,[x := e_x]\textbf{\texttt{;}}
}
\]

\vspace{10mm}
\textbf{Binary expression}: All occurrences of $x$ in the operands are substituted with $e_x$.
\[
\frac{
}{
  (e_1 \  \textit{binary-operator} \ e_2)[x := e_x] 
  \ = \ 
  e_1[x := e_x] \ \textit{binary-operator} \ e_2[x := e_x]
}
\]

\vspace{10mm}
\textbf{Unary expression}: All occurrences of $x$ in the operand are substituted with $e_x$.
\[
\frac{
}{
  (\textit{unary-operator} \ e)[x := e_x]
  \ = \ 
  \textit{unary-operator} \ e[x := e_x]
}
\]

\vspace{10mm}
\textbf{Conditional expression}: All occurrences of $x$ in the operands are substituted with $e_x$.
\[
\frac{
}{
  (e_1\  \textbf{\texttt{?}}\ e_2\ \textbf{\texttt{:}}\ e_3)[x := e_x]
  \ = \ 
  e_1[x := e_x]\  \textbf{\texttt{?}}\ e_2[x := e_x]\ \textbf{\texttt{:}}\ e_3[x := e_x]
}
\]

\vspace{10mm}
\textbf{Logical expression}: All occurrences of $x$ in the operands are substituted with $e_x$.
\[
\frac{
}{
  (e_1\  \textbf{\texttt{||}}\ e_2)[x := e_x]
  \ = \ 
  e_1[x := e_x]\  \textbf{\texttt{||}}\ e_2[x := e_x]
}
\]

\[
\frac{
}{
  (e_1\  \textbf{\texttt{\&\&}}\ e_2)[x := e_x]
  \ = \ 
  e_1[x := e_x]\  \textbf{\texttt{\&\&}}\ e_2[x := e_x]
}
\]

\vspace{10mm}
\textbf{Call expression}: All occurrences of $x$ in the arguments and the function expression of the application $e$ are substituted with $e_x$.
\[
\frac{
}{
  (e \ \textbf{\texttt{(}}\ x_1 \ldots x_n \ \textbf{\texttt{)}})[x := e_x]
  \ = \ 
  e[x := e_x] \ \textbf{\texttt{(}}\ x_1[x := e_x] \ldots x_n[x := e_x] \ \textbf{\texttt{)}}
}
\]

\textbf{Function declaration}: All occurrences of $x$ in the body of a function are substituted with $e_x$ under given circumstances.
\begin{enumerate}
    \item Function declaration where $x$ has the same name as a parameter.

    \[
    \frac{
      \exists \, i \in \{1, \cdots, n\} \text{ s.t. } x \ = \ x_i
    }{
      (\textbf{\texttt{function}}\  \textit{name} \ \textbf{\texttt{(}}\ x_1 \ldots x_n \ \textbf{\texttt{)}}\ \textit{block})[x := e_x]
      \ = \ 
      \textbf{\texttt{function}}\  \textit{name} \ \textbf{\texttt{(}}\ x_1 \ldots x_n \ \textbf{\texttt{)}}\ \textit{block}
    }
    \]
    \item Function declaration where $x$ does not have the same name as a parameter.
    \begin{enumerate}
        \item No parameter of the function occurs free in $e_x$.
        \[
        \frac{
          \forall \, i \in \{1, \cdots, n\} \text{ s.t. } x \ \neq \ x_i \text{, } \ \forall \, j \in \{1, \cdots, n\} \text{ s.t. } x_j \text{ does not occur free in $e_x$}
        }{
          \substack{\substack{\displaystyle{(\textbf{\texttt{function}}\  \textit{name} \ \textbf{\texttt{(}}\ x_1 \ldots x_n \ \textbf{\texttt{)}}\ \textit{block})[x := e_x]} \vspace{1.5mm} \\ \vspace{0.5mm} \displaystyle{=}} \\  \displaystyle{ \textbf{\texttt{function}}\  \textit{name} \ \textbf{\texttt{(}}\ x_1 \ldots x_n \ \textbf{\texttt{)}}\ \textit{block}[x := e_x]}}
        }
        \]
        \item A parameter of the function occurs free in $e_x$.
        \[
        \frac{
          \forall \, i \in \{1, \cdots, n\} \text{ s.t. } x \ \neq \ x_i \text{, } \ \exists \, j \in \{1, \cdots, n\} \text{ s.t. } x_j \text{ occurs free in $e_x$}
        }{
          \substack{\substack{\displaystyle{(\textbf{\texttt{function}}\  \textit{name} \ \textbf{\texttt{(}}\ x_1 \ldots x_j \ldots x_n \ \textbf{\texttt{)}}\ \textit{block})[x := e_x]} \vspace{1.5mm} \\ \vspace{0.5mm} \displaystyle{=}} \\  \displaystyle{(\textbf{\texttt{function}}\  \textit{name} \ \textbf{\texttt{(}}\ x_1 \ldots y \ldots x_n \ \textbf{\texttt{)}}\ \textit{block}[x_j := y])[x := e_x]}}
        }
        \]
        Substitution is applied to the whole expression again as to recursively detect and rename all parameters of the function declaration that clash with variables that occur free in $e_x$, at which point (\,i\,) takes place. Note that the name $y$ is not declared in, nor occurs in \textit{block} and $e_x$.
    \end{enumerate}
    \end{enumerate}

\vspace{10mm}
\textbf{Lambda expression}: All occurrences of $x$ in the body of a lambda expression are substituted with $e_x$ under given circumstances.
\begin{enumerate}
    \item Lambda expression where $x$ has the same name as a parameter.
    \[
    \frac{
      \exists \, i \in \{1, \cdots, n\} \text{ s.t. } x \ = \ x_i
    }{
      (\textbf{\texttt{(}}\ x_1 \ldots x_n \ \textbf{\texttt{)}}\ \textbf{\texttt{=>}}\ \textit{block})[x := e_x]
      \ = \ 
      \textbf{\texttt{(}}\ x_1 \ldots x_n \ \textbf{\texttt{)}}\ \textbf{\texttt{=>}}\ \textit{block}
    }
    \]
    \item Lambda expression where $x$ does not have the same name as a parameter.
    \begin{enumerate}
        \item No parameter of the lambda expression occurs free in $e_x$.
        \[
        \frac{
          \forall \, i \in \{1, \cdots, n\} \text{ s.t. } x \ \neq \ x_i \text{, } \ \forall \, j \in \{1, \cdots, n\} \text{ s.t. } x_j \text{ does not occur free in $e_x$}
        }{
          (\textbf{\texttt{(}}\ x_1 \ldots x_n \ \textbf{\texttt{)}}\ \textbf{\texttt{=>}}\ \textit{block})[x := e_x]
          \ = \ 
          \textbf{\texttt{(}}\ x_1 \ldots x_n \ \textbf{\texttt{)}}\ \textbf{\texttt{=>}}\ \textit{block}[x := e_x]
        }
        \]
        \item A parameter of the lambda expression occurs free in $e_x$.
        \[
        \frac{
          \forall \, i \in \{1, \cdots, n\} \text{ s.t. } x \ \neq \ x_i \text{, } \ \exists \, j \in \{1, \cdots, n\} \text{ s.t. } x_j \text{ occurs free in $e_x$}
        }{
          (\textbf{\texttt{(}}\ x_1 \ldots x_j \ldots x_n \ \textbf{\texttt{)}}\ \textbf{\texttt{=>}}\ \textit{block})[x := e_x]
          \ = \ 
          (\textbf{\texttt{(}}\ x_1 \ldots y \ldots x_n \ \textbf{\texttt{)}}\ \textbf{\texttt{=>}}\ \textit{block}[x_j := y])[x := e_x]
        }
        \]
        Substitution is applied to the whole expression again as to recursively detect and rename all parameters of the lambda expression that clash with variables that occur free in $e_x$, at which point (\,i\,) takes place. Note that the name $y$ is not declared in, nor occurs in \textit{block} and $e_x$.
        \end{enumerate}
        \end{enumerate}

\pagebreak
\textbf{Block expression}: All occurrences of $x$ in the statements of a block expression are substituted with $e_x$ under given circumstances.
\begin{enumerate}
    \item Block expression in which $x$ is declared.
    \[
    \frac{
      x \text{ is declared in \textit{block}}
    }{
      \textit{block}[x := e_x]
      \ = \ 
      \textit{block}
    }
    \]
    \item Block expression in which $x$ is not declared.
    \begin{enumerate}
        \item No names declared in the block occurs free in $e_x$.
        \[
        \frac{
           x \text{ is not declared in \textit{block}}\text{, } \ \textit{name} \text{ declared in \textit{block} does not occur free in } e_x
        }{
          \textit{block}[x := e_x]
          \ = \ 
          [\textit{block}[0][x := e_x] \text{, } \ldots \text{, } \textit{block}[n][x := e_x]]
        }
        \]
        \item A name declared in the block occurs free in $e_x$.
        \[
        \frac{
          x \text{ is not declared in \textit{block}}\text{, } \ \textit{name} \text{ declared in \textit{block} occurs free in } e_x
        }{
          \textit{block}[x := e_x]
          \ = \ 
          [\textit{block}[0][\textit{name} := y] \text{, } \ldots \text{, } \textit{block}[n][\textit{name} := y]][x := e_x]
        }
        \]
        Substitution is applied to the whole expression again as to recursively detect and rename all declared names of the block expression that clash with variables that occur free in $e_x$, at which point (\,i\,) takes place. Note that the name $y$ is not declared in, nor occurs in \textit{block} and $e_x$.
        \end{enumerate}
        \end{enumerate}

\vspace{10mm}
\textbf{Variable declaration}: All occurrences of $x$ in the declarators of a variable declaration are substituted with $e_x$.
\[
\frac{
}{
  \textit{declarations}[x := e_x]
  \ = \ 
  [\textit{declarations}[0][x := e_x] \ldots \textit{declarations}[n][x := e_x]]
}
\]

\vspace{10mm}
\textbf{Return statement}: All occurrences of $x$ in the expression that is to be returned are substituted with $e_x$.
\[
\frac{
}{
  (\textbf{\texttt{return}} \ e\textbf{\texttt{;}})[x := e_x]
  \ = \ 
  \textbf{\texttt{return}} \ e[x := e_x]\textbf{\texttt{;}}
}
\]

\vspace{10mm}
\textbf{Conditional statement}: All occurrences of $x$ in the condition, consequent, and alternative expressions of a conditional statement are substituted with $e_x$.
\[
\frac{
}{
  (
  \textbf{\texttt{if}}\ 
  \textbf{\texttt{(}}\ 
  e \ 
  \textbf{\texttt{)}} \ 
  \textit{block} \
  \textbf{\texttt{else}} \ 
  \textit{block}
  )[x := e_x]
  \ = \ 
  \textbf{\texttt{if}}\ 
  \textbf{\texttt{(}}\ 
  e[x := e_x] \ 
  \textbf{\texttt{)}} \ 
  \textit{block}[x := e_x] \
  \textbf{\texttt{else}} \ 
  \textit{block}[x := e_x]
}
\]

\vspace{10mm}
\textbf{Array expression}: All occurrences of $x$ in the elements of an array are substituted with $e_x$.
\[
\frac{
}{
  [x_1 \text{, } \ldots \text{, } x_n][x := e_x]
  \ = \ 
  [x_1[x := e_x] \text{, } \ldots \text{, } x_n[x := e_x]]
}
\]

\pagebreak
\section*{Free names}

Let $\rhd$ be the relation that defines the set of free names of a given Source expression; the symbols $p_1$ and $p_2$ shall henceforth refer to unary and binary operations, respectively. That is, $p_1$ ranges over $\{\textbf{\texttt{!}}\}$ and $p_2$ ranges over $\{\textbf{\texttt{||}},\, \textbf{\texttt{\&\&}},\, \textbf{\texttt{+}},\, \textbf{\texttt{-}},\, \textbf{\texttt{*}},\, \textbf{\texttt{/}},\, \textbf{\texttt{===}},\, \textbf{\texttt{>}},\, \textbf{\texttt{<}}\}$.

\vspace{10mm}
\textbf{Identifier}:
\[
\frac{
}{  
  x
  \ \rhd \
  \{x\}
}
\]
\[
\frac{
}{  
  \textit{name}
  \ \rhd \
  \varnothing
}
\]

\vspace{10mm}
\textbf{Boolean}:
\[
\frac{
}{  
  \textbf{\texttt{true}}
  \ \rhd \
  \varnothing
}
\]
\[
\frac{
}{  
  \textbf{\texttt{false}}
  \ \rhd \
  \varnothing
}
\]

\vspace{10mm}
\textbf{Expression statement}:
\[
\frac{
  e \ \rhd \ S
}{  
  e \textbf{\texttt{;}} \ \rhd \ S
}
\]

\vspace{10mm}
\textbf{Unary expression}:
\[
\frac{
  e \ \rhd \ S
}{  
  p_1(e) \ \rhd \ S
}
\]

\vspace{10mm}
\textbf{Binary expression}:
\[
\frac{
  e_1 \ \rhd \ S_1
  \text{, } \
  e_2 \ \rhd \ S_2
}{  
  p_1(e_1, e_2)
  \ \rhd \
  S_1 \cup S_2
}
\]

\vspace{10mm}
\textbf{Conditional expression}:
\[
\frac{
  e_1 \ \rhd \ S_1
  \text{, } \
  e_2 \ \rhd \ S_2
  \text{, } \
  e_3 \ \rhd \ S_3
}{  
  e_1\  \textbf{\texttt{?}}\ e_2\ \textbf{\texttt{:}}\ e_3
  \ \rhd \
  S_1 \cup S_2 \cup S_3
}
\]

\vspace{10mm}
\textbf{Call expression}:
\[
\frac{
  e \ \rhd \ S
  \text{, } \
  e_k \ \rhd \ T_k
}{  
  e(e_1,\, \ldots,\, e_n)
  \ \rhd \
  S \cup T_1 \cup \ldots \cup T_n
}
\]

\vspace{10mm}
\textbf{Function declaration}:
\[
\frac{
  \textit{block} \ \rhd \ S
}{  
  \textbf{\texttt{function}}\  \textit{name} \ \textbf{\texttt{(}}\ x_1 \ldots x_n \ \textbf{\texttt{)}}\ \textit{block}
  \ \rhd \
  S - \{x_1,\, \ldots,\, x_n\}
}
\]

\vspace{10mm}
\textbf{Lambda expression}:
\[
\frac{
  \textit{block} \ \rhd \ S
}{  
  \textbf{\texttt{(}}\ x_1 \ldots x_n \ \textbf{\texttt{)}}\ \textbf{\texttt{=>}}\ \textit{block}
  \ \rhd \
  S - \{x_1,\, \ldots,\, x_n\}
}
\]

\vspace{10mm}
\textbf{Block expression}:
\[
\frac{
  \textit{block}[k] \ \rhd \ S_k
  \text{, } \
  T \text{ contains all names declared in \textit{block}}
}{  
  \textit{block}
  \ \rhd \
  (S_1 \cup \ldots \cup S_n) - T
}
\]

\vspace{10mm}
\textbf{Constant declaration}:
\[
\frac{
  e \ \rhd \ S
}{  
  \textbf{\texttt{const}}\  \textit{name} \ 
             \textbf{\texttt{=}}\ e\textbf{\texttt{;}}
  \ \rhd \
  S
}
\]

\vspace{10mm}
\textbf{Return statement}:
\[
\frac{
  e \ \rhd \ S
}{  
  \textbf{\texttt{return}}\ e\textbf{\texttt{;}}
  \ \rhd \
  S
}
\]

\vspace{10mm}
\textbf{Conditional statement}:
\[
\frac{
  e \ \rhd \ S
  \text{, } \
  \textit{block}_1 \ \rhd \ T_1
  \text{, } \
  \textit{block}_2 \ \rhd \ T_2
}{  
  \textbf{\texttt{if}}\ 
  \textbf{\texttt{(}}\ 
  e \ 
  \textbf{\texttt{)}} \ 
  \textit{block}_1 \
  \textbf{\texttt{else}} \ 
  \textit{block}_2
  \ \rhd \
  S \cup T_1 \cup T_2
}
\]


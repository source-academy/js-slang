\subsection*{List Support}

The following list processing functions are supported:

\begin{itemize}
\item \lstinline{pair(x, y)}: \textit{primitive}, makes a pair from \lstinline{x} and \lstinline{y}; time: $\Theta(1)$, space: $\Theta((1)$.
\item \lstinline{is_pair(x)}: \textit{primitive}, returns \lstinline{true} if \lstinline{x} is a
  pair and \lstinline{false} otherwise; time: $\Theta(1)$, space: $\Theta((1)$.
\item \lstinline{head(x)}: \textit{primitive}, returns the head (first component) of the pair \lstinline{x}; time: $\Theta(1)$, space: $\Theta((1)$.
\item \lstinline{tail(x)}: \textit{primitive}, returns the tail (second component) of the
  pair \lstinline{x}; time: $\Theta(1)$, space: $\Theta((1)$.
\item \lstinline{is_null(xs)}: \textit{primitive}, returns \lstinline{true} if \lstinline{xs} is the
  empty list \lstinline{null}, and \lstinline{false} otherwise; time: $\Theta(1)$, space: $\Theta((1)$.
\item \lstinline{is_list(x)}: \textit{primitive}, returns \lstinline{true} if
  \lstinline{x} is a list as defined in the lectures, and
  \lstinline{false} otherwise. Iterative process; 
time: $\Theta(n)$, space: $\Theta(1)$, where $n$ is the length of the 
chain of \lstinline{tail} operations that can be applied to \lstinline{x}.
\item \lstinline{list(x1, x2,..., xn)}: \textit{primitive}, returns a list with $n$ elements. The
first element is \lstinline{x1}, the second \lstinline{x2}, etc. Iterative
process; time: $\Theta(n)$, space: $\Theta(n)$, since the constructed list data structure
consists of $n$ pairs, each of which takes up a constant amount of space.
\item \lstinline{draw_data(x1, x2,..., xn)}: \textit{primitive}, visualizes each \lstinline{x1, x2,..., xn} in a separate drawing
  area in the Source Academy using a box-and-pointer diagram; time, space:
  $\Theta(n)$, where $n$ is the combined number of data structures such as
  pairs in \lstinline{x1, x2,..., xn}.
\item \lstinline{equal(x1, x2)}: Returns \lstinline{true} if both
  have the same structure with respect to \lstinline{pair},
  and the same numbers, boolean values, functions or empty list
  at corresponding leave positions (places that are not themselves pairs),
  and \lstinline{false} otherwise; time, space:
  $\Theta(n)$, where $n$ is the number of data structures such as pairs
  in \lstinline{x1} and \lstinline{x2}.
\item \lstinline{length(xs)}: Returns the length of the list
  \lstinline{xs}. 
Iterative process; time: $\Theta(n)$, space: $\Theta(1)$, where $n$ is the length of \lstinline{xs}.
\item \lstinline{map(f, xs)}: Returns a list that results from list
  \lstinline{xs} by element-wise application of \lstinline{f}. 
Iterative process; time: $\Theta(n)$ (apart from \lstinline{f}), space: $\Theta(n)$ (apart from \lstinline{f}), where $n$ is the length of \lstinline{xs}.
\item \lstinline{build_list(f, n)}: Makes a list with \lstinline{n}
elements by applying the unary function \lstinline{f} to the numbers 0 to \lstinline{n - 1}.
Iterative process; time: $\Theta(n)$ (apart from \lstinline{f}), space: $\Theta(n)$ (apart from \lstinline{f}).
\item \lstinline{for_each(f, xs)}: Applies \lstinline{f} to every
  element of the list \lstinline{xs}, and then returns
  \lstinline{true}. 
Iterative process; time: $\Theta(n)$ (apart from \lstinline{f}), space: $\Theta(1)$ (apart from \lstinline{f}), where $n$ is the length of \lstinline{xs}.
\item \lstinline{list_to_string(xs)}: Returns a string that represents
list \lstinline{xs} using the text-based box-and-pointer notation \lstinline{[...]}.
\item \lstinline{reverse(xs)}: Returns list \lstinline{xs} in reverse
  order. Iterative process; time: $\Theta(n)$, space: $\Theta(n)$, where $n$ is the length of \lstinline{xs}.
The process is iterative, but consumes space $\Theta(n)$ because of the result list.
\item \lstinline{append(xs, ys)}: Returns a list that results from 
appending the list \lstinline{ys} to the list \lstinline{xs}.
Iterative process; time: $\Theta(n)$, space: $\Theta(n)$, where $n$ is the length of \lstinline{xs}.
\item \lstinline{member(x, xs)}: Returns first postfix sublist
whose head is identical to
\lstinline{x} (\lstinline{===}); returns \lstinline{[]} if the
element does not occur in the list.
Iterative process; time: $\Theta(n)$, space: $\Theta(1)$, where $n$ is the length of \lstinline{xs}.
\item \lstinline{remove(x, xs)}: Returns a list that results from
\lstinline{xs} by removing the first item from \lstinline{xs} that
is identical (\lstinline{===}) to \lstinline{x}. Iterative process;
time: $\Theta(n)$, space: $\Theta(n)$, where $n$ is the length of \lstinline{xs}.
\item \lstinline{remove_all(x, xs)}: Returns a list that results from
\lstinline{xs} by removing all items from \lstinline{xs} that
are identical (\lstinline{===}) to \lstinline{x}.
Iterative process;
time: $\Theta(n)$, space: $\Theta(n)$, where $n$ is the length of \lstinline{xs}.
\item \lstinline{filter(pred, xs)}: Returns a list that contains
only those elements for which the one-argument function
\lstinline{pred}
returns \lstinline{true}.
Iterative process;
time: $\Theta(n)$ (apart from \lstinline{pred}), space: $\Theta(n)$ (apart from \lstinline{pred}), where $n$ is the length of \lstinline{xs}.
\item \lstinline{enum_list(start, end)}: Returns a list that enumerates
numbers starting from \lstinline{start} using a step size of 1, until
the number exceeds (\lstinline{>}) \lstinline{end}.
Iterative process;
time: $\Theta(n)$, space: $\Theta(n)$, where $n$ is \lstinline{end} $-$  \lstinline{start}.
\item \lstinline{list_ref(xs, n)}: Returns the element
of list \lstinline{xs} at position \lstinline{n}, 
where the first element has index 0.
Iterative process;
time: $\Theta(n)$ (apart from \lstinline{f}), space: $\Theta(1)$ (apart from \lstinline{f}), where $n$ is the length of \lstinline{xs}.
\item \lstinline{accumulate(f, initial, xs)}: Applies binary
function \lstinline{f} to the elements of \lstinline{xs} from
right-to-left order, first applying \lstinline{f} to the last element
and the value \lstinline{initial}, resulting in $r_1$, then to the 
second-last element and $r_1$, resulting in $r_2$, etc, and finally
to the first element and $r_{n-1}$, where $n$ is the length of the
list. Thus, \lstinline{accumulate(f,zero,list(1,2,3))} results in
\lstinline{f(1, f(2, f(3, zero)))}.
Iterative process;
time: $\Theta(n)$, space: $\Theta(n)$, where $n$ is the length of \lstinline{xs},
assuming \lstinline{f} takes constant time.
\end{itemize}

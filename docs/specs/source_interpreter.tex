\section*{Interpreter Support}

\begin{itemize}
\item \lstinline{apply_in_underlying_javascript(f, xs)}: \textit{primitive}, calls the function \lstinline{f}
with arguments \lstinline{xs}. For example:
\begin{lstlisting}
function times(x, y) {
   return x * y;
}
apply_in_underlying_javascript(times, list(2, 3)); // returns 6
\end{lstlisting}
\item \lstinline{tokenize(x)}: \textit{primitive}, returns the list of tokens that results from tokenizing
  the string \lstinline{x} as a Source program. Each token is a string that contains the characters of
  the token as they appear in the program. Comments are ignored.
\item \lstinline{parse(x)}: \textit{primitive}, returns the parse tree that results from parsing
  the string \lstinline{x} as a Source program. The following two pages describe the shape of the parse tree.
  The tree is represented by the tagged lists on the right; the angle brackets denote recursive application
  of the transformation rules. Implementations are allowed to support more of JavaScript than listed.
\end{itemize}
In addition, the Source Academy frontend predeclares the name \lstinline{__PROGRAM__} in all Source languages to
refer to the string representation of the entrypoint file of the program in the editor that is being run using ``Run''.
The entrypoint file is the file containing the code that acts as the entrypoint of the program being run.
If \lstinline{__PROGRAM__} is used in the REPL, it refers to the string representation of the editor content
at the time when ``Run'' was last pressed.

\newpage
\KOMAoptions{paper=landscape,pagesize}
\recalctypearea
	\addtolength{\oddsidemargin}{-5cm}
	\addtolength{\evensidemargin}{-5cm}
	\addtolength{\textwidth}{13cm}

\begin{alignat*}{9}
&& \textit{program}    &&\quad ::= &\quad && \textit{statement} \ \ldots
                                                           && \texttt{list("sequence", list of  } \langle \textit{statement}\rangle \textit{)} \\[1mm]
&& \textit{statement}    &&\quad ::= &\quad && \textbf{\texttt{const}}\  \textit{name} \
                                           \textbf{\texttt{=}}\  \textit{expression} \ \textbf{\texttt{;}}
                                                           && \texttt{list("constant\_declaration",  } \langle \textit{name}\rangle \textit{,  } \langle \textit{expression}\rangle \textit{)} \\
&&                       && |   &\quad && \textit{let} \ \textbf{\texttt{;}}
                                                           &&  \textrm{see below}\\
&&                       && |   &\quad && \textbf{\texttt{function}}\  \textit{name} \
                                   \textbf{\texttt{(}}\  \textit{parameters} \ \textbf{\texttt{)}}\ \textit{block} \quad
                                                           &&  \texttt{list("function\_declaration",  } \langle \textit{name}\rangle \textit{, } \langle \textit{parameters}\rangle \textit{,}\ \langle \textit{block}\rangle \textit{)} \\
&&                       && |   &\quad && \textbf{\texttt{return}}\  \textit{expression} \ \textbf{\texttt{;}}
                                                           && \texttt{list("return\_statement",  } \langle \textit{expression}\rangle \textit{)} \\
&&                       && |   &\quad && \textit{if-statement} \quad
                                                           && \textrm{see below}\\
&&                       && |   &\quad && \textbf{\texttt{while}}\
                                   \textbf{\texttt{(}}\  \textit{expression} \ \textbf{\texttt{)}} \
                                   \textit{block}
                                                           && \texttt{list("while\_loop",  } \langle \textit{expression}\rangle \textit{,  } \langle \textit{block}\rangle \textit{)} \\
&&                       && |   &\quad && \textbf{\texttt{for}}\ \textbf{\texttt{(}} \
                                          (\ \textit{\hyperlink{for}{expression}}_1 \ | \  \textit{\hyperlink{for2}{let}}\ ) \textbf{\texttt{;}} \\
&&                       &&     &\quad && \ \ \ \ \ \ \ \ \ \ \textit{expression}_2 \ \textbf{\texttt{;}} && \texttt{list("for\_loop",  } \langle \textit{expression}_1\rangle\ \textrm{or}\ \langle \textit{let}\rangle \textit{,  } \langle \textit{expression}_2\rangle \textit{,  } \langle \textit{expression}_3\rangle \textit{,}\\
&&                       &&     &\quad && \ \ \ \ \ \ \ \ \ \ \textit{expression}_3 \ \textbf{\texttt{)}} \  \textit{block}
                                            && \texttt{ }\ \ \ \ \ \ \ \langle \textit{block}\rangle \textit{)} \\
&&                       && |   &\quad && \textbf{\texttt{break}}\ \textbf{\texttt{;}}
                                                           && \texttt{list("break\_statement")} \\
&&                       && |   &\quad && \textbf{\texttt{continue}}\ \textbf{\texttt{;}}
                                                           && \texttt{list("continue\_statement")} \\
&&                       && |   &\quad &&  \textit{block}
                                                           && \textrm{see below}\\
%&&                       && |   &\quad &&  \textbf{\texttt{try}}\ \textit{block}_1 \ \textbf{\texttt{catch (}}\ \textit{name}\ \textbf{\texttt{)}}\ \textit{block}_2
%                                                           && \texttt{list("try\_statement",}\ \langle  \textit{block}_1 \rangle \texttt{,}\ \langle  \textit{name} \rangle \texttt{,}\ \langle  \textit{block}_2 \rangle \texttt{)} \\
%&&                       && |   &\quad &&  \textbf{\texttt{throw}}\ \textit{expression}
%                                                           && \texttt{list("throw\_statement",}\ \langle  \textit{expression} \rangle \texttt{)} \\
&&                       && |   &\quad &&  \textit{expression} \ \textbf{\texttt{;}}
                                                           && \textrm{see below}\\[1mm]
&& \textit{parameters}   && ::= &\quad &&  \epsilon\ | \  \textit{name} \
                                                   (\ \textbf{\texttt{,}} \ \textit{name}\ )\ \ldots
                                                            && \texttt{list of  } \langle \textit{name}\rangle  \\
&& \textit{if-statement} && ::= &\quad &&  \textbf{\texttt{if}}\
                                   \textbf{\texttt{(}}\ \textit{expression} \ \textbf{\texttt{)}}\
                                   \textit{block}_1 \\
&&                       &&     &      && \textbf{\texttt{else}}\
                                          (\ \textit{block}_2
                                          \ | \
                                          \textit{\href{https://sourceacademy.org/sicpjs/1.3.3\#footnote-1}{if-statement}} \ )
                                          && \texttt{list("conditional\_statement",  } \langle \textit{expression}\rangle \textit{, } \\
                                            &&&&&&&&&\ \ \ \ \ \ \texttt{ } \langle \textit{block}_1\rangle \textit{,  } \langle \textit{block}_2\rangle \ \textrm{or}\ \langle \textit{if-statement} \rangle \texttt{)} \\
&& \textit{block}        && ::= &      && \textbf{\texttt{\{}}\  \textit{program}   \ \textbf{\texttt{\}}} \quad
                                                           && \texttt{list("block",  } \langle \textit{program}\rangle \textit{)} \\
&& \textit{let}          && ::= &\quad &&  \textbf{\texttt{let}}\  \textit{name} \
                                           \textbf{\texttt{=}}\  \textit{expression}
                                                            && \texttt{list("variable\_declaration",  } \langle \textit{name}\rangle \textit{,  } \langle \textit{expression}\rangle \textit{)} \\
&& \textit{assignment}   && ::= &\quad &&  \textit{name} \
                                           \textbf{\texttt{=}}\  \textit{expression}
                                                            && \texttt{list("assignment",  } \langle \textit{name}\rangle \textit{,  } \langle \textit{expression}\rangle \textit{)} \\
&&                       && |   &\quad && \textit{expression}_1 \textbf{\texttt{[}}
                                          \textit{expression}_2 \textbf{\texttt{]}} \
                                           \textbf{\texttt{=}}\  \textit{expression}_3  \textbf{\texttt{;}}\
                                                           && \texttt{list("object\_assignment", } \langle \textit{expression}_1 \textbf{\texttt{[}} \textit{expression}_2 \textbf{\texttt{]}}  \rangle \textit{,  } \langle \textit{expression}_3\rangle \textit{)}
\end{alignat*}

\begin{alignat*}{9}
  && \textit{expression}
                         &&\quad  ::= &\quad && \textit{number}  && \texttt{list("literal",}\ \textit{number}\texttt{)} \\
&&                       && |   &\quad && \textbf{\texttt{true}}
                                                           && \texttt{list("literal",}\ \textbf{\texttt{true}}\texttt{)} \\
&&                       && |   &\quad && \textbf{\texttt{false}}
                                                           && \texttt{list("literal",}\ \textbf{\texttt{false}}\texttt{)} \\
&&                       && |   &\quad && \textbf{\texttt{null}}
                                                           && \texttt{list("literal",}\ \textbf{\texttt{null}}\texttt{)} \\
&&                       && |   &\quad &&  \textit{string}
                                                           && \texttt{list("literal",}\ \textit{string}\texttt{)} \\
&&                       && |   &\quad &&  \textit{name}   && \texttt{list("name", string)}  \\
&&                       && |   &\quad &&  \textit{expression}_1 \  \textit{log-op} \
                                            \textit{expression}_2 \qquad
                                                           && \texttt{list("logical\_composition",  } \langle \textit{log-op}\rangle \textit{, } \langle \textit{expression}_1\rangle \textit{, } \langle \textit{expression}_2\rangle \textit{)} \\
&&                       && |   &\quad &&  \textit{expression}_1 \  \textit{bin-op} \
                                            \textit{expression}_2 \qquad
                                                           && \texttt{list("binary\_operator\_combination",  } \langle \textit{bin-op}\rangle \textit{, } \langle \textit{expression}_1\rangle \textit{, } \langle \textit{expression}_2\rangle \textit{)} \\
&&                       && |   &\quad &&   \textit{un-op} \
                                            \textit{expression}
                                                           && \texttt{list("unary\_operator\_combination",  } \langle \textit{un-op}\rangle \textit{, } \langle \textit{expression}\rangle \textit{)} \\
&&                       && |   &\quad &&   \textit{expression} \
                                            \textbf{\texttt{(}}\ \textit{expressions}\
                                            \textbf{\texttt{)}}
                                                           && \texttt{list("application",  } \langle \textit{expression}\rangle \textit{, list of  } \langle \textit{expression}\rangle \textit{)} \\
&&                       && |   &\quad &&   (\ \textit{name}\ | \
                                               \textbf{\texttt{(}}\ \textit{parameters}\ \textbf{\texttt{)}}\
                                            )\
                                            \texttt{\textbf{=>}}\ \textit{expression}
                                            && \texttt{list("lambda\_expression",  } \langle \textit{parameters}\rangle \textit{,}  \\
                                              && && & && && \texttt{list("return\_statement",  } \langle \textit{expression}\rangle \textit{))} \\
&&                       && |   &\quad &&   (\ \textit{name}\ | \
                                               \textbf{\texttt{(}}\ \textit{parameters}\ \textbf{\texttt{)}}\
                                            )\
                                            \texttt{\textbf{=>}}\ \textit{block}
                                                           && \texttt{list("lambda\_expression",  } \langle \textit{parameters}\rangle \textit{,  } \langle \textit{statement}\rangle \textit{)} \\
&&                       && |   &\quad &&   \textit{expression}_1 \ \textbf{\texttt{?}}\
                                            \textit{expression}_2
                                            \ \textbf{\texttt{:}}\
                                            \textit{expression}_3\
                                                           && \texttt{list("conditional\_expression",  } \langle \textit{expression}_1\rangle \textit{,} \\
                                            &&&&&&&&&\ \ \ \ \ \ \texttt{ } \langle \textit{expression}_2\rangle \textit{,  } \langle \textit{expression}_3\rangle \textit{)} \\
&&                       && |   &\quad && \textit{assignment} \ \textbf{\texttt{;}}
                                                           && \\
&&                       && |   &\quad && \textit{expression}_1 \textbf{\texttt{[}}
                                          \textit{expression}_2 \textbf{\texttt{]}}
                                                           && \texttt{list("object\_access",  } \langle \textit{expression}_1\rangle \textit{,  } \langle \textit{expression}_2\rangle \textit{)} \\
&&                       && |   &\quad &&   \textbf{\texttt{[}}\
                                            \textit{expressions}\
                                            \textbf{\texttt{]}}
                                                           && \texttt{list("array\_expression", list of  } \langle \textit{expression}\rangle \textit{)} \\
% &&                       && |   &\quad &&  \textbf{\texttt{new}}\ \textit{expression}
%                                                           && \texttt{list("new\_expression",}\ \langle  \textit{expression} \rangle \texttt{)} \\
%&&                       && |   &\quad &&  \textit{expression}\ \textbf{\texttt{.}}\ \textit{name}
%                                                           && \textrm{treat as}:\ \textit{expression}\ \textbf{\texttt{[ "}}\textit{name}\textbf{\texttt{" ]}}  \\
&&                       && |   &\quad &&  \textbf{\texttt{(}}\  \textit{expression} \
                                            \textbf{\texttt{)}} && \textrm{treat as:}\ \textit{expression} \\[1mm]
&& \textit{log-op}
                        &&\quad  ::= &\quad && \textbf{\texttt{\&\&}}\ |\ \texttt{\textbf{||}}
                                          && \textit{string representing operator} \\[1mm]
&& \textit{bin-op}
                        &&\quad  ::= &\quad && \textbf{\texttt{+}}\ |\ \textbf{\texttt{-}}\ |\ \textbf{\texttt{*}}\ |\ \textbf{\texttt{/}}\ |\ \textbf{\texttt{\%}}\ |\
                                   \textbf{\texttt{===}}\ |\ \textbf{\texttt{!==}}  && \textit{string representing operator} \\
&&                       && |  &\quad &&  \texttt{\textbf{<}}\ |\ \texttt{\textbf{>}}\ |\ \texttt{\textbf{<=}}\ |\ \texttt{\textbf{>=}}
                                          && \textit{string representing operator} \\[1mm]
&& \textit{un-op}
                        &&\quad  ::= &\quad && \textbf{\texttt{!}}
                        && \texttt{"!"} \\
&&
                        && | &\quad && \textbf{\texttt{-}}
                        && \texttt{"-unary"} \\
&& \textit{expressions}  &&\quad  ::= &\quad && \epsilon\ | \ \textit{expression}\ (
                                                               \ \textbf{\texttt{,}} \
                                                                 \textit{expression} \
                                                                      )\ \ldots
                                                            && \texttt{list of  } \langle \textit{expression}\rangle  \\
\end{alignat*}



\newpage
\KOMAoptions{paper=portrait}
\recalctypearea

% ALIGN EVEN- AND ODD-NUMBERED PAGES.
\evensidemargin 35pt

% HORIZONTAL MARGINS
% Left margin 1 inch (0 + 1)
\setlength{\oddsidemargin}{0in}
% Text width 6.5 inch (so right margin 1 inch).
\setlength{\textwidth}{6.5in}

% ----------------
% VERTICAL MARGINS
% Top margin 0.5 inch (-0.5 + 1)
\setlength{\topmargin}{-0.5in}
% Head height 0.25 inch (where page headers go)
\setlength{\headheight}{0.25in}
% Head separation 0.25 inch (between header and top line of text)
\setlength{\headsep}{0.25in}
% Text height 8.5 inch (so bottom margin 1.5 in)
\setlength{\textheight}{10.0in}



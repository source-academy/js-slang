\section*{Names}

Names\footnote{
In
\href{http://www.ecma-international.org/publications/files/ECMA-ST/Ecma-262.pdf}{
\color{DarkBlue}ECMAScript 2018 ($9^{\textrm{th}}$ Edition)},
these names are called \emph{identifiers}.
} start with \verb@_@, \verb@$@ or a
letter\footnote{
By \emph{letter}
we mean \href{http://unicode.org/reports/tr44/}{\color{DarkBlue}Unicode} letters (L) or letter numbers (NI).
} and contain only \verb@_@, \verb@$@,
letters or digits\footnote{
By \emph{digit} we mean characters in the
\href{http://unicode.org/reports/tr44/}{Unicode} categories
Nd (including the decimal digits 0, 1, 2, 3, 4, 5, 6, 7, 8, 9), Mn, Mc  and Pc. 
}. Reserved words\footnote{
By \emph{Reserved word} we mean any of:
$\textbf{\texttt{break}}$, $\textbf{\texttt{case}}$, $\textbf{\texttt{catch}}$, $\textbf{\texttt{continue}}$, $\textbf{\texttt{debugger}}$, $\textbf{\texttt{default}}$, $\textbf{\texttt{delete}}$, $\textbf{\texttt{do}}$, $\textbf{\texttt{else}}$, $\textbf{\texttt{finally}}$, $\textbf{\texttt{for}}$, $\textbf{\texttt{function}}$, $\textbf{\texttt{if}}$, $\textbf{\texttt{in}}$, $\textbf{\texttt{instanceof}}$, $\textbf{\texttt{new}}$, $\textbf{\texttt{return}}$, $\textbf{\texttt{switch}}$, $\textbf{\texttt{this}}$, $\textbf{\texttt{throw}}$, $\textbf{\texttt{try}}$, $\textbf{\texttt{typeof}}$, $\textbf{\texttt{var}}$, $\textbf{\texttt{void}}$, $\textbf{\texttt{while}}$, $\textbf{\texttt{with}}$, $\textbf{\texttt{class}}$, $\textbf{\texttt{const}}$, $\textbf{\texttt{enum}}$, $\textbf{\texttt{export}}$, $\textbf{\texttt{extends}}$, $\textbf{\texttt{import}}$, $\textbf{\texttt{super}}$, $\textbf{\texttt{implements}}$, $\textbf{\texttt{interface}}$, $\textbf{\texttt{let}}$, $\textbf{\texttt{package}}$, $\textbf{\texttt{private}}$, $\textbf{\texttt{protected}}$, $\textbf{\texttt{public}}$, $\textbf{\texttt{static}}$, $\textbf{\texttt{yield}}$, $\textbf{\texttt{null}}$, $\textbf{\texttt{true}}$, $\textbf{\texttt{false}}$.
} such as keywords are not allowed as names.

Valid names are \verb@x@, \verb@_45@, \verb@$$@ and $\mathtt{\pi}$,
but always keep in mind that programming is communicating and that the familiarity of the
audience with the characters used in names is an important aspect of program readability.

In addition to names that
are declared using \texttt{\textbf{const}}, \texttt{\textbf{function}},
$\texttt{\textbf{=>}}$ (and \texttt{\textbf{let}} in Source \S3 and 4), the following
names refer to primitive functions and constants:
\begin{itemize}
\item \href{https://sicp.comp.nus.edu.sg/chapters/6\#footnote-link-4}{\lstinline{math_}$\textit{name}$},
where $\textit{name}$ is any name specified in the
JavaScript
\texttt{Math} library, see\\
\href{https://www.ecma-international.org/ecma-262/9.0/index.html\#sec-math-object}{\color{DarkBlue}ECMAScript Specification, Section 20.2}. Examples:
\begin{itemize}
\item \verb#math_PI#: Refers to the mathematical constant $\pi$,
\item \verb#math_sqrt#\texttt{(n)}: Returns the square root of the \emph{number} \texttt{n}.
\end{itemize}
\item \texttt{\href{https://sicp.comp.nus.edu.sg/chapters/17\#ex_1.22}{runtime()}}: Returns number of milliseconds elapsed since January 1, 1970 00:00:00 UTC
\item \verb#parse_int#\texttt{(s, i)}:
interprets the \emph{string} \texttt{s} as an integer, using the positive integer \texttt{i} as radix, and returns the respective value,
see \href{https://www.ecma-international.org/ecma-262/9.0/index.html\#sec-parseint-string-radix}{\color{DarkBlue}ECMAScript Specification, Section 18.2.5}.
\item \href{https://sicp.comp.nus.edu.sg/chapters/4}{\texttt{undefined}},
  \texttt{\href{https://www.ecma-international.org/ecma-262/9.0/index.html\#sec-value-properties-of-the-global-object-nan}{\color{DarkBlue}NaN}}, \texttt{\href{https://www.ecma-international.org/ecma-262/9.0/index.html\#sec-value-properties-of-the-global-object-infinity}{\color{DarkBlue}Infinity}}: Refer to JavaScript's undefined,
NaN (``Not a Number'') and Infinity values, respectively.
\item \verb#is_boolean#\texttt{(x)}, \href{https://sicp.comp.nus.edu.sg/chapters/36}{\lstinline{is_number(x)}},
  \href{https://sicp.comp.nus.edu.sg/chapters/36}{\lstinline{is_string(x)}}, \verb#is_function#\texttt{(x)}:
        return \texttt{true} if the type of \texttt{x} matches the function name and \texttt{false} if it does not. Following
        JavaScript, we specify that \verb#is_number# returns \texttt{true} for \texttt{NaN} and \texttt{Infinity}.
\item \texttt{prompt(s)}: Pops up a window that displays the \emph{string} \texttt{s}, provides
an input line for the user to enter a text, a ``Cancel'' button and an ``OK'' button. The call of \texttt{prompt}
suspends execution of the program until one of the two buttons is pressed. If 
the ``OK'' button is pressed, \texttt{prompt} returns the entered text as a string.
If the ``Cancel'' button is pressed, \texttt{prompt} returns a non-string value.
\item \href{https://sicp.comp.nus.edu.sg/chapters/17\#footnote-6}{\texttt{display(x)}}: Displays the value \texttt{x} in the console\footnote{The notation used for the display of values is consistent with \href{http://www.ecma-international.org/publications/files/ECMA-ST/ECMA-404.pdf}{\color{DarkBlue}JSON}, but also displays \texttt{undefined} and function objects.}; returns the argument \texttt{a}.
\item \texttt{display(x, s)}: Displays the string \texttt{s}, followed by a space character, followed by the value \texttt{x} in the console\footnotemark[\value{footnote}]; returns the argument \texttt{x}.
\item \href{https://sicp.comp.nus.edu.sg/chapters/21\#footnote-3}{\texttt{error(x)}}: Displays the value \texttt{x} in the console\footnotemark[\value{footnote}] with error flag. The evaluation
  of any call of \texttt{error} aborts the running program immediately.
\item \texttt{error(x, s)}: Displays the string \texttt{s}, followed by a space character, followed by the value \texttt{x} in the console\footnotemark[\value{footnote}] with error flag. The evaluation
  of any call of \texttt{error} aborts the running program immediately.
\item \href{https://sicp.comp.nus.edu.sg/chapters/62}{\lstinline{stringify(x)}}: returns a string that represents\footnotemark[\value{footnote}] the value \texttt{x}. 
\end{itemize}
All Source primitive functions, except \verb#stringify#, can be assumed to run
in $O(1)$ time, except \texttt{display}, \texttt{error} and \texttt{stringify}, 
which run in $O(n)$ time, where $n$ is
the size (number of components such as pairs) of their argument.

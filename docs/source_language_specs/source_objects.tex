\section*{Objects}

\subsection*{Object properties}

Literal objects can be created using literal object expressions:
\begin{lstlisting}
let my_obj = { "key 1": 1,
               "key 2": 2 };
\end{lstlisting}

As keys, only strings are allowed in Source.
If the string (without quotation marks) looks like a \textit{name}, the
quotation marks around property keys can be omitted, for example:
\begin{lstlisting}
let the_motorcycle = { cc: 399,
                      number_of_cylinders: 4 };
\end{lstlisting}

The syntax
\begin{lstlisting}
my_obj["key 1"];
the_motorcycle["cc"] = 896;
\end{lstlisting}
allows for object access and assignment. Here, the quotation marks are not optional,
even when the string (without quotation marks) looks like a \textit{name}.

\subsection*{Dot Abbreviation}

In $\textit{statement}$ and $\textit{expression}$, the syntax
\[
\textit{expression} \ \texttt{.}\ \textit{id}
\]
can be used as an abbreviation for
\[
\textit{expression} \ \textbf{\texttt{[}}\  \texttt{\bf "} \textit{id}
\texttt{\bf "}\ \textbf{\texttt{]}}
\]
if the string $\textit{id}$ looks like a \textit{name}.
For example
\begin{lstlisting}
my_motorcyle . number_of_cylinders;
my_motorcycle . cc = 1249;
\end{lstlisting}
are abbreviations for
\begin{lstlisting}
my_motorcyle [ "number_of_cylinders" ];
my_motorcycle [ "cc" ] = 1249;
\end{lstlisting}

\documentclass[epic,eepic,10pt,a4paper]{article}
\usepackage[usenames,dvipsnames,svgnames,table]{xcolor}
\usepackage{graphicx,bookman}
\usepackage{graphics}
\usepackage{CJKutf8}
\usepackage{latexsym}

\usepackage{amsmath}

\usepackage[T1]{fontenc}
\usepackage{listings}

\lstdefinelanguage{JavaScript}{
  keywords={const, let, break, case, catch, continue, debugger, default, delete, do, else, finally, for, function, if, in, instanceof, new, return, switch, this, throw, try, typeof, var, void, while, with},
  morecomment=[l]{//},
  morecomment=[s]{/*}{*/},
  morestring=[b]',
  morestring=[b]",
  sensitive=true
}

\lstset{
   language=JavaScript,
   basicstyle=\ttfamily,
   showstringspaces=false,
   showspaces=false,
   escapechar={^}
}

\pagestyle{myheadings}

% ALIGN EVEN- AND ODD-NUMBERED PAGES.
\evensidemargin 35pt

% HORIZONTAL MARGINS
% Left margin 1 inch (0 + 1)
\setlength{\oddsidemargin}{0in}
% Text width 6.5 inch (so right margin 1 inch).
\setlength{\textwidth}{6.5in}

% ----------------
% VERTICAL MARGINS
% Top margin 0.5 inch (-0.5 + 1)
\setlength{\topmargin}{-0.5in}
% Head height 0.25 inch (where page headers go)
\setlength{\headheight}{0.25in}
% Head separation 0.25 inch (between header and top line of text)
\setlength{\headsep}{0.25in}
% Text height 8.5 inch (so bottom margin 1.5 in)
\setlength{\textheight}{10.0in}

% ----------------
% PARAGRAPH INDENTATION
\setlength{\parindent}{0in}

% SPACE BETWEEN PARAGRAPHS
%\setlength{\parskip}{\medskipamount}

% ----------------
% EVALUATION SYMBOL
\newcommand{\evalsto}{$\Longrightarrow$}

% ----------------
% STRUTS
% HORIZONTAL STRUT.  One argument (width).
\newcommand{\hstrut}[1]{\hspace*{#1}}
% VERTICAL STRUT. Two arguments (offset from baseline, height).
\newcommand{\vstrut}[2]{\rule[#1]{0in}{#2}}

% ----------------
% EMPTY BOXES OF VARIOUS WIDTHS, FOR INDENTATION
\newcommand{\hm}{\hspace*{1em}}
\newcommand{\hmm}{\hspace*{2em}}
\newcommand{\hmmm}{\hspace*{3em}}
\newcommand{\hmmmm}{\hspace*{4em}}

% ----------------
% VARIOUS CONVENIENT WIDTHS RELATIVE TO THE TEXT WIDTH, FOR BOXES.
\newlength{\hlessmm}
\setlength{\hlessmm}{\textwidth}
\addtolength{\hlessmm}{-2em}

\newlength{\hlessmmmm}
\setlength{\hlessmmmm}{\textwidth}
\addtolength{\hlessmmmm}{-4em}

% ----------------
% ``TIGHTLIST'' ENVIRONMENT (no para space between items, small indent)
\newenvironment{tightlist}%
{\begin{list}{$\bullet$}{%
    \setlength{\topsep}{0in}
    \setlength{\partopsep}{0in}
    \setlength{\itemsep}{0in}
    \setlength{\parsep}{0in}
    \setlength{\leftmargin}{1.5em}
    \setlength{\rightmargin}{0in}
    \setlength{\itemindent}{0in}
}
}%
{\end{list}
}

% ----------------
% CODE FONT (e.g. {\cf x := 0}).
\newcommand{\cf}{\footnotesize\tt}

% ----------------
% INSTRUCTION POINTER
\newcommand{\IP}{$\bullet$}
\newcommand{\goesto}{$\longrightarrow$}

% \illuswidth is used to set up boxes around illustrations.
\newlength{\illuswidth}
\setlength{\illuswidth}{\textwidth}
\addtolength{\illuswidth}{-7pt}

% PROBLEM SET HEADER -- args are semester and problem set or solution
% example: \psetheader{Spring Semester, 1989}{Problem set 1}
\newcommand{\docheader}[3]{%
\markright{SICP, JavaScript Adaptation, #2 #3, #1}
\begin{center}
\Large
National University of Singapore \\
School of Computing \\
Martin Henz \\
\medskip
    {\Large {\bf #2 #3}, #1}\\
    \medskip
{\large \today}
\end{center}
}

% ----------------------------------------------------------------
% HERE BEGINS THE DOCUMENT
% start with \begin{document}


        \usepackage{url}
        \usepackage{hyperref}

\newcommand{\qed}{$\Box$}
\newcommand{\Rule}[2]{\genfrac{}{}{0.7pt}{}{{\setlength{\fboxrule}{0pt}\setlength{\fboxsep}{3mm}\fbox{$#1$}}}{{\setlength{\fboxrule}{0pt}\setlength{\fboxsep}{3mm}\fbox{$#2$}}}}
\newcommand{\Rulee}[3]{\genfrac{}{}{0.7pt}{}{{\setlength{\fboxrule}{0pt}\setlength{\fboxsep}{3mm}\fbox{$#1$}}}{{\setlength{\fboxrule}{0pt}\setlength{\fboxsep}{3mm}\fbox{$#2$}}}[#3]}
\newcommand{\transition}{\rightrightarrows_s}
\newcommand{\translate}{\twoheadrightarrow}
\newcommand{\translateaux}{\hookrightarrow}
\begin{document}
\subsection{Adding Arrays}
Arrays in SVML are Javascript arrays.

In order to implement arrays, we add \texttt{NEWA} (New array),
\texttt{LDAG} (Load boxed value from array),
\texttt{STAG} (Store boxed value into array) and \texttt{DUP}
(Duplicate top of stack) to our instruction set. For this section,
we will be using 0-indexing to refer to the elements instead for clarity.

\paragraph{Compilation and Execution of Array Literals}
Array literals of size $n$ explicitly declares all $n$ elements of the array.

An array literal of size $0$ explicitly declares $0$ elements of the array, and so does not explicitly declare any elements of the array.
So, an array literal of size $0$ compiles to the machine instruction that declares a new empty array, \texttt{NEWA}.

$\Rule{}{\lbrack\rbrack \translateaux \texttt{NEWA}}$

When executed, \texttt{NEWA} allocates a heap node to the array, in which contains the Javascript newly declared empty array, and then puts an array reference, $\lbrack\rbrack$, to this heap node at the top of the operand stack.

\textit{(A note on notation: $\lbrack\rbrack$ is written as such to remind us of the state of the Javascript array in the referenced heap node. This notation is purely for convenience of reading.
Although this notation represents the Javascript array, it really is just a reference to the heap node containing the Javascript array.
Since $\lbrack\rbrack$ does not contain anything, it represents the fact that the Javascript array has length $0$.
)}

So, \texttt{NEWA} follows this execution rule:

$\Rule{s(\textit{pc}) = \texttt{NEWA}}{
(\textit{os},\textit{pc},e,\textit{rs}) \transition
(\lbrack\rbrack.os,\textit{pc}+1,e,\textit{rs})}
$

An array literal of size $n$, where $n$ is not $0$, does two things: it declares a new array, as well as stores each of its declared elements into the array.

So, this non-empty array literal compiles as such: to \texttt{NEWA}, and to some other instructions (\texttt{STAG}, \texttt{DUP}, and \texttt{LDCI}) to store its declared elements into the array. Storing elements into the array involves these extra instructions because the machine state needs to be managed, due to the particular way \texttt{STAG} is executed.

\texttt{STAG} stores an element at an index in an array. It pops three operands from the operand stack: the element to be stored, $y$, the index to store the element in, $i$, and the array reference, $\lbrack\cdots\rbrack$. The execution rule is:

\textit{(A note on notation: we represent an array of unknown length, and unknown elements, with $\lbrack\cdots\rbrack$.
Elements at index $j$ are denoted with $a_j$, for $j \in \mathbb{N}$ (since the array has unknown length), regardless of whether the value of $a_j$ is defined.
However, if we know that the array has an element of value $v$ at index $i$, then the array is represented as: $\lbrack\cdots v \cdots\rbrack$. We then explicitly say that $v$ is at index $i$.
)}

$\Rule{s(\textit{pc}) = \texttt{STAG}}{
(v.i.\lbrack\cdots\rbrack.\textit{os},\textit{pc},e,\textit{rs}) \\
\transition (\textit{os},\textit{pc}+1,e,\textit{rs})}
$

where after execution, the array is $\lbrack\cdots v \cdots\rbrack$, where $v$ is at index $i$.

The particular way \texttt{STAG} is executed can be seen in the rule: after execution, the array reference is no longer on the operand stack.
This works fine for array literals of size $1$, as seen by the rules:

Compilation, where the $0^{th}$ element is stored into the $0^{th}$ index of the array:

$\Rule{E_0 \translateaux s_0}{
\lbrack E_0 \rbrack \translateaux \texttt{NEWA}
.\texttt{LDCI}\ 0.s_0.\texttt{STAG}}
$

Execution, where

$\begin{aligned}
& s(\textit{pc}) = \texttt{NEWA}\\
& s(\textit{pc+1}) = \texttt{LDCI}\ 0\\
& s(\textit{pc+2}) = s_0\\
& s(\textit{pc+3}) = \texttt{STAG}\\
\end{aligned}$

we have:

$\Rule{s(\textit{pc}) = \texttt{NEWA}}{
(\textit{os},\textit{pc},e,\textit{rs}) \transition
(\lbrack\rbrack.\textit{os},\textit{pc}+1,e,\textit{rs})}
$

$\Rule{s(\textit{pc}+1) = \texttt{LDCI}\ 0}{
(\lbrack\rbrack.\textit{os},\textit{pc}+1,e,\textit{rs}) \transition
(0.\lbrack\rbrack.\textit{os},\textit{pc}+2,e,\textit{rs})}
$

$\Rule{s(\textit{pc}+2) = s_0}{
(0.\lbrack\rbrack.\textit{os},\textit{pc}+2,e,\textit{rs}) \transition
(v.0.\lbrack\rbrack.\textit{os},\textit{pc}+3,e,\textit{rs})}
$

where $s_0$ executes to have $v$ on top of the operand stack.

$\Rule{s(\textit{pc}+3) = \texttt{STAG}}{
(v.0.\lbrack\rbrack.\textit{os},\textit{pc}+3,e,\textit{rs}) \transition
(\textit{os},\textit{pc}+4,e,\textit{rs})}
$

where after execution, the array is $\lbrack v \rbrack$, where $v$ is at index $0$.

\textit{(A note on notation: here, since we know that the array has length $1$ and an element of value $v$ at index $0$, we do not explicitly write the $\cdots$, and so the array is represented as: $\lbrack v \rbrack$. This is to avoid confusion with standard array notation.
)}

However, this becomes problematic for array literals of size $n > 1$, as seen by the rules:

Compilation:

$\Rule{E_0 \translateaux s_0 \qquad E_1 \translateaux s_1}{
\lbrack E_0, E_1 \rbrack \translateaux \texttt{NEWA}
.\texttt{LDCI}\ 0.s_0.\texttt{STAG}
.\texttt{LDCI}\ 1.s_1.\texttt{STAG}}
$

Execution, where

$\begin{aligned}
& s(\textit{pc}) = \texttt{NEWA}\\
& s(\textit{pc+1}) = \texttt{LDCI}\ 0\\
& s(\textit{pc+2}) = s_0\\
& s(\textit{pc+3}) = \texttt{STAG}\\
& s(\textit{pc+4}) = \texttt{LDCI}\ 1\\
& s(\textit{pc+5}) = s_1\\
& s(\textit{pc+6}) = \texttt{STAG}\\
\end{aligned}$

and execution proceeds as before, continuing with:

$\Rule{s(\textit{pc}+4) = \texttt{LDCI}\ 1}{
(\textit{os},\textit{pc}+4,e,\textit{rs}) \transition
(1.\textit{os},\textit{pc}+5,e,\textit{rs})}
$

$\Rule{s(\textit{pc}+5) = s_1}{
(1.\textit{os},\textit{pc}+5,e,\textit{rs}) \transition
(v.1.\textit{os},\textit{pc}+6,e,\textit{rs})}
$

where $s_1$ executes to have $v$ on top of the operand stack.

Here is the problem: execution cannot continue:

$\Rule{s(\textit{pc}+6) = \texttt{STAG}}{
(v.1.\textit{os},\textit{pc}+6,e,\textit{rs}) \transition \cdots}
$

because there is no array reference on the operand stack for \texttt{STAG} to pop.

The solution is to compile array literals of size $n$, where $n$ is not $0$, with the \texttt{DUP} instruction.
$\texttt{DUP}$ pops the value at the top of the operand stack
and pushes the value back twice.

$\Rule{s(\textit{pc}) = \texttt{DUP}}{
(x.\textit{os},\textit{pc},e,\textit{rs}) \transition
(x.x.\textit{os},\textit{pc}+1,e,\textit{rs})}
$

The rule for compiling array literals of size $n$, where $n$ is not $0$, is thus:

$\Rule{E_0 \translateaux s_0 \qquad \cdots \qquad E_n \translateaux s_n}{
\lbrack E_0, ..., E_n\rbrack \translateaux s_0. \texttt{NEWA}
.\texttt{DUP}.\texttt{LDCI}\ 0.s_0.\texttt{STAG}. \cdots
.\texttt{DUP}.\texttt{LDCI}\ n.s_n\texttt{STAG}}
$

where for each element of index $i$ in the array, we add the machine instructions
$\texttt{DUP}.\texttt{LDCI}\ i.s_i.\texttt{STAG}$.

\paragraph{Compilation and Execution of Array Assignment}
We used \texttt{STAG} in the compilation of array literals
above. \texttt{STAG} is similarly used for typical storing of elements in
arrays through assignments.

$\Rule{
E_0 \translateaux s_0 \qquad E_1 \translateaux s_1 \qquad E_2 \translateaux s_2}
{E_0\lbrack E_1\rbrack = E_2 \translateaux s_0.s_1.s_2.\texttt{STAG}}
$

Execution works the same as it did above. Note that the new element will replace
the element that was stored at the same index in the array previously, if any.
\texttt{STAG} will also update the length of the array if the index to
store the element in is larger than any previously used index for this array.

\paragraph{Compilation and Execution of Array Access}

Lastly, to load an element from an array at an index,
and push the element to the top of the operand stack, we use \texttt{LDAG}.

$\Rule{
E_0 \translateaux s_0 \qquad E_1 \translateaux s_1}
{E_0\lbrack E_1\rbrack \translateaux s_0.s_1.\texttt{LDAG}}
$

As for execution, if there is the array does contain an element in the
corresponding index, it will load the element as such:

$\Rule{s(\textit{pc}) = \texttt{LDAG} \qquad a_i \text{ is assigned}}{
(i.\lbrack\cdots a_i\cdots\rbrack.\textit{os},\textit{pc},e,\textit{rs}) \transition
(a_i.\textit{os},\textit{pc}+1,e,\textit{rs})}
$

Otherwise, it will load \texttt{undefined} instead, like in the case below:

$\Rule{s(\textit{pc}) = \texttt{LDAG} \qquad a_i \text{ is not assigned}}{
(i.\lbrack\cdots\rbrack.\textit{os},\textit{pc},e,\textit{rs}) \transition
(\texttt{undefined}.\textit{os},\textit{pc}+1,e,\textit{rs})}
$
\end{document}
\documentclass[epic,eepic,10pt,a4paper]{article}
\usepackage[usenames,dvipsnames,svgnames,table]{xcolor}
\usepackage{graphicx,bookman}
\usepackage{graphics}
\usepackage{CJKutf8}
\usepackage{latexsym}

\usepackage{amsmath}

\usepackage[T1]{fontenc}
\usepackage{listings}

\lstdefinelanguage{JavaScript}{
  keywords={const, let, break, case, catch, continue, debugger, default, delete, do, else, finally, for, function, if, in, instanceof, new, return, switch, this, throw, try, typeof, var, void, while, with},
  morecomment=[l]{//},
  morecomment=[s]{/*}{*/},
  morestring=[b]',
  morestring=[b]",
  sensitive=true
}

\lstset{
   language=JavaScript,
   basicstyle=\ttfamily,
   showstringspaces=false,
   showspaces=false,
   escapechar={^}
}

\pagestyle{myheadings}

% ALIGN EVEN- AND ODD-NUMBERED PAGES.
\evensidemargin 35pt

% HORIZONTAL MARGINS
% Left margin 1 inch (0 + 1)
\setlength{\oddsidemargin}{0in}
% Text width 6.5 inch (so right margin 1 inch).
\setlength{\textwidth}{6.5in}

% ----------------
% VERTICAL MARGINS
% Top margin 0.5 inch (-0.5 + 1)
\setlength{\topmargin}{-0.5in}
% Head height 0.25 inch (where page headers go)
\setlength{\headheight}{0.25in}
% Head separation 0.25 inch (between header and top line of text)
\setlength{\headsep}{0.25in}
% Text height 8.5 inch (so bottom margin 1.5 in)
\setlength{\textheight}{10.0in}

% ----------------
% PARAGRAPH INDENTATION
\setlength{\parindent}{0in}

% SPACE BETWEEN PARAGRAPHS
%\setlength{\parskip}{\medskipamount}

% ----------------
% EVALUATION SYMBOL
\newcommand{\evalsto}{$\Longrightarrow$}

% ----------------
% STRUTS
% HORIZONTAL STRUT.  One argument (width).
\newcommand{\hstrut}[1]{\hspace*{#1}}
% VERTICAL STRUT. Two arguments (offset from baseline, height).
\newcommand{\vstrut}[2]{\rule[#1]{0in}{#2}}

% ----------------
% EMPTY BOXES OF VARIOUS WIDTHS, FOR INDENTATION
\newcommand{\hm}{\hspace*{1em}}
\newcommand{\hmm}{\hspace*{2em}}
\newcommand{\hmmm}{\hspace*{3em}}
\newcommand{\hmmmm}{\hspace*{4em}}

% ----------------
% VARIOUS CONVENIENT WIDTHS RELATIVE TO THE TEXT WIDTH, FOR BOXES.
\newlength{\hlessmm}
\setlength{\hlessmm}{\textwidth}
\addtolength{\hlessmm}{-2em}

\newlength{\hlessmmmm}
\setlength{\hlessmmmm}{\textwidth}
\addtolength{\hlessmmmm}{-4em}

% ----------------
% ``TIGHTLIST'' ENVIRONMENT (no para space between items, small indent)
\newenvironment{tightlist}%
{\begin{list}{$\bullet$}{%
    \setlength{\topsep}{0in}
    \setlength{\partopsep}{0in}
    \setlength{\itemsep}{0in}
    \setlength{\parsep}{0in}
    \setlength{\leftmargin}{1.5em}
    \setlength{\rightmargin}{0in}
    \setlength{\itemindent}{0in}
}
}%
{\end{list}
}

% ----------------
% CODE FONT (e.g. {\cf x := 0}).
\newcommand{\cf}{\footnotesize\tt}

% ----------------
% INSTRUCTION POINTER
\newcommand{\IP}{$\bullet$}
\newcommand{\goesto}{$\longrightarrow$}

% \illuswidth is used to set up boxes around illustrations.
\newlength{\illuswidth}
\setlength{\illuswidth}{\textwidth}
\addtolength{\illuswidth}{-7pt}

% PROBLEM SET HEADER -- args are semester and problem set or solution
% example: \psetheader{Spring Semester, 1989}{Problem set 1}
\newcommand{\docheader}[3]{%
\markright{SICP, JavaScript Adaptation, #2 #3, #1}
\begin{center}
\Large
National University of Singapore \\
School of Computing \\
Martin Henz \\
\medskip
    {\Large {\bf #2 #3}, #1}\\
    \medskip
{\large \today}
\end{center}
}

% ----------------------------------------------------------------
% HERE BEGINS THE DOCUMENT
% start with \begin{document}


        \usepackage{url}
        \usepackage{hyperref}

\newcommand{\qed}{$\Box$}
\newcommand{\Rule}[2]{\genfrac{}{}{0.7pt}{}{{\setlength{\fboxrule}{0pt}\setlength{\fboxsep}{3mm}\fbox{$#1$}}}{{\setlength{\fboxrule}{0pt}\setlength{\fboxsep}{3mm}\fbox{$#2$}}}}
\newcommand{\Rulee}[3]{\genfrac{}{}{0.7pt}{}{{\setlength{\fboxrule}{0pt}\setlength{\fboxsep}{3mm}\fbox{$#1$}}}{{\setlength{\fboxrule}{0pt}\setlength{\fboxsep}{3mm}\fbox{$#2$}}}[#3]}
\newcommand{\transition}{\rightrightarrows_s}
\newcommand{\translate}{\twoheadrightarrow}
\newcommand{\translateaux}{\hookrightarrow}
\newcommand{\Lc}{\texttt{\{}}
\newcommand{\Rc}{\texttt{\}}}
\begin{document}
\subsection{Adding Loops}

In Source §3, there are two kinds of loops, while loops and for loops.
For SVML, to simplify compilation for loops, we transform all for loops into while loops as
indicated in the Source §3 specifications. Therefore, we will only cover
the compilation and execution of while loops.

\paragraph{Compilation and Execution of While Loops}
While loops have two components, the predicate, and the body. As long as the
predicate evaluates to true, the machine should execute the statements in the body.
Therefore, compilation simply uses the \texttt{BRF} instruction to skip
the instructions in the body when the predicate evaluates to false.
In addition, we also include a \texttt{BR} instruction, to branch back
the start to evaluate the predicate again.

$\Rule{
E_1 \translateaux s_1 \qquad S \translateaux s}
{\textit{while}\ (E_1)\ \Lc\ S\ \Rc\ \translateaux s_1
.\texttt{BRF}\ |s|+3.\texttt{NEWENV}.s.\texttt{POPENV}
.\texttt{BR}\ -(|s|+|s_1|+3)}
$

Execution follows the rules mentioned in the previous sections.

\paragraph{Compilation and Execution of Break statements}
Break statements simply exit the loop and continue execution from there.
We likewise use a \texttt{BR} instruction for this, where the offset
is the displacement in \textit{pc} from the break instruction to the end
of the loop. We include a
\texttt{POPENV} instruction to remove the environment created by loop.

$\Rule{}
{\textit{break} \translateaux \texttt{POPENV}.\texttt{BR}\ x}
$

\paragraph{Compilation and Execution of Continue statements}
Continue statements skip to the next iteration of the loop. For while loops,
we can just branch to the end of the last statement of the body. However, for
for loops, as we compile them to while loops, there is an additional
increment statement that needs to be executed after each iteration. Therefore,
to ensure for loops are supported, we branch to the instruction which increments
the for loop counter instead, which is the last statement in the body of the while loop.
Therefore, the offset is displacement in \textit{pc} from the
continue instruction to the start of the last statement of the body for for loops,
and the continue instruction to the end of the last statement of the body for while loops.

$\Rule{}
{\textit{continue} \translateaux \texttt{BR}\ x}
$
\end{document}
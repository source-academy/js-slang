\section*{Interpreter Support}

\begin{itemize}
\item \lstinline{is_number(x)}: \textit{builtin}, returns \lstinline{true} if \lstinline{x} is a number, and 
\lstinline{false} otherwise.
\item \lstinline{is_boolean(x)}: \textit{builtin}, returns \lstinline{true} if \lstinline{x} is \lstinline{true} or \lstinline{false}, and \lstinline{false} otherwise.
\item \lstinline{is_string(x)}: \textit{builtin}, returns \lstinline{true} if \lstinline{x} is a
string, and \lstinline{false} otherwise.
\item \lstinline{is_function(x)}: \textit{builtin}, returns \lstinline{true} if \lstinline{x} is a
function, and \lstinline{false} otherwise.
\item \lstinline{is_object(x)}: \textit{builtin}, returns \lstinline{true} if \lstinline{x} is an
object, and \lstinline{false} otherwise. Following JavaScript, arrays are considered
objects.
\item \lstinline{is_array(x)}: \textit{builtin}, returns \lstinline{true} if \lstinline{x} is an
array, and \lstinline{false} otherwise. The empty array \lstinline{[]}, also known
as the empty list, is an array.
\item \lstinline{parse(x)}: \textit{builtin}, returns the parse tree that results from parsing
the string \lstinline{x} as a Source program.
\item \lstinline{JSON.stringify(x)}: \textit{builtin}, returns a string that represents the given JSON object
\lstinline{x}.
\item \lstinline{apply_in_underlying_javascript(f, xs)}: \textit{builtin}, calls the function \lstinline{f}
with arguments \lstinline{xs}. For example:
\begin{lstlisting}
function times(x, y) {
   return x * y;
}
apply_in_underlying_javascript(times, list(2, 3)); // returns 6
\end{lstlisting}
\end{itemize}

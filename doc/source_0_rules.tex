\section*{Reduction}

The 
\emph{program reducer} $\Rrightarrow$ is a partial function from
programs to
programs
and $\Rrightarrow^*$ is its reflexive transitive closure.
Within the definition of $\Rrightarrow$, we make use of
the following two groups of analogous partial functions:
The statement
\emph{reducer} $\Rightarrow$ is a partial function from
statements to statements
and $\Rightarrow^*$ is its reflexive transitive closure.
The expression
\emph{reducer} $\rightarrow$ is a partial function from
expressions to expressions
and $\rightarrow^*$ is its reflexive transitive closure.

A \emph{program reduction} is a sequence of programs
$p_1 \Rrightarrow \cdots \Rrightarrow p_n$,
where $p_n$ is not reducible, i.e. there is no
program $q$ such that $p_n \Rightarrow q$.
Here, the program $p_n$ is called the \emph{result
of reducing} $p_1$.

A \emph{value} is a primitive number expression,
primitive boolean expression,
a primitive string expression, a function definition
expression or a function declaration statement.

The \emph{substitution} function 
$p [ n \leftarrow v ]$ on programs/statements/expressions
replaces every free occurrence of the name $n$
in statement $p$ by value $v$. Care must be taken to introduce
and preserve
co-references in this process; substitution can introduce
cyclic references in the result of the substitution. For example,
$n$ may occur free in $v$, in which case
every occurrence of $n$ in $p$
will be replaced by $v$ such that $n$ in $v$ refers cyclically
to the node at which the replacement happens.


\subsection*{How to read inference rules}

To read the rules below, think of each "fraction" as an
"if we know X then we can infer Y" implication statement.
For example,

\[
\frac{
  X = 5
}{
  X + 1 = 5 + 1
}
\]

corresponds to the statement
"If we know $X$ is equal to $5$,
then we can infer that $X + 1$ is equal to $5 + 1$".


\subsection*{Programs}


\textbf{First-statement}: In a sequence of statements, we can always
reduce the first one.
\[
\frac{
  \textit{statement}\ \Rightarrow\ \textit{statement}'
}{  
  \textit{statement} \ldots
  \Rrightarrow 
  \textit{statement}' \ldots
}
\]

\vspace{10mm}

\textbf{Eliminate-function-declaration}: Function declarations as first
statements are substituted in the remaining statements.
\[
\frac{
             f = \textbf{\texttt{function}}\  \textit{name} \ 
                 \textbf{\texttt{(}}\  \textit{parameters}
                 \ \textbf{\texttt{)}}\ \textit{block}
}{
f\ \textit{statement} \ldots\ 
  \Rrightarrow\ 
  \textit{statement} \ldots[\textit{name} \leftarrow f]
}
\]

\vspace{10mm}
\textbf{Eliminate-Values}: Values as first statemments are discarded, if
they are preceding one or more statements in a statement sequence.
\[
\frac{
v \ \mbox{is a value}  
}{
v \textbf{\texttt{;}} \textit{statement}+\ 
   \Rrightarrow  \ 
  \textit{statement}+
}
\]

\subsection*{Statements: Blocks}

\textbf{Block-statement-reduce}: A block statement is
reducible if its program is reducible.
\[
\frac{
  \textit{program} 
\ \Rrightarrow \ 
  \textit{program}'
}{  
  \textbf{\texttt{\{}} \
  \textit{program} \ 
  \textbf{\texttt{\}}}
\  \Rightarrow \ 
  \textbf{\texttt{\{}} \
  \textit{program}' \ 
  \textbf{\texttt{\}}}
}
\]

\subsection*{Statements: Expression statements}

\textbf{Expression-statement-reduce}: An expression statement
is reducible if its expression is reducible.
\[
\frac{
  e\ \rightarrow\ e'
}{  
  e \textbf{\texttt{;}}
  \ \Rightarrow \ 
  e' \textbf{\texttt{;}}
}
\]


\subsection*{Expressions: Blocks}

\textbf{Block-expression-reduce}: A block expression is
reducible if its program is reducible.
\[
\frac{
  \textit{program} 
\ \Rrightarrow \ 
  \textit{program}'
}{  
  \textbf{\texttt{\{}} \
  \textit{program} \ 
  \textbf{\texttt{\}}}
\  \rightarrow \ 
  \textbf{\texttt{\{}} \
  \textit{program}' \ 
  \textbf{\texttt{\}}}
}
\]

\vspace{10mm}

\textbf{Block-expression-undefined}: A block expression
whose body only contains a single value statement reduces to
the value 
\textbf{\texttt{undefined}}.

\[
\frac{
}{  
  \textbf{\texttt{\{}} \
  v \ 
  \textbf{\texttt{;}} \ 
  \textbf{\texttt{\}}}
\  \rightarrow \ 
\textbf{\texttt{undefined}}
}
\]

\textbf{Block-expression-reduce-return}: A block expression
whose body only contains a single return statement can 
be reduced by reducing the return expression.

\[
\frac{
  e
\ \rightarrow \ 
  e'
}{  
  \textbf{\texttt{\{}} \
  \textbf{\texttt{return}} \ e \ 
  \textbf{\texttt{;}} \ 
  \textbf{\texttt{\}}}
\  \rightarrow \ 
  \textbf{\texttt{\{}} \
  \textbf{\texttt{return}} \ e' \ 
  \textbf{\texttt{;}} \ 
  \textbf{\texttt{\}}}
}
\]

\vspace{10mm}

\textbf{Block-expression-eliminate-return}: A block expression
whose body only contains a single return value reduces to
that value.

\[
\frac{
}{  
  \textbf{\texttt{\{}} \
  \textbf{\texttt{return}} \ v \ 
  \textbf{\texttt{;}} \ 
  \textbf{\texttt{\}}}
\  \rightarrow \ 
  v
}
\]


\subsection*{Expressions: Binary operators}

\textbf{Left-binary-reduce}: An expression with binary operator
can be reduced if its left sub-expression can be reduced.
\[
\frac{
  e_1 \ \rightarrow \ e_1'
}{
  e_1\  \textit{binary-operator} \ e_2
  \ \rightarrow \ 
  e_1'\  \textit{binary-operator} \ e_2
}
\]


\vspace{10mm}
\textbf{And-shortcut-false}: An expression with binary operator
$\textbf{\texttt{\&\&}}$ whose left sub-expression is
$\textbf{\texttt{false}}$ can be reduced to
$\textbf{\texttt{false}}$.
\[
\frac{
}{
  \textbf{\texttt{false}}\  \textbf{\texttt{\&\&}}\ e
  \ \rightarrow \ 
  \textbf{\texttt{false}}
}
\]

\vspace{10mm}
\textbf{And-shortcut-true}: An expression with binary operator
$\textbf{\texttt{\&\&}}$ whose left sub-expression is
$\textbf{\texttt{true}}$ can be reduced to
the right sub-expression.
\[
\frac{
}{
  \textbf{\texttt{true}}\  \textbf{\texttt{\&\&}}\ e
  \ \rightarrow \ 
  e
}
\]

\vspace{10mm}
\textbf{Or-shortcut-true}: An expression with binary operator
$\textbf{\texttt{||}}$ whose left sub-expression is
$\textbf{\texttt{true}}$ can be reduced to
$\textbf{\texttt{true}}$.
\[
\frac{
}{
  \textbf{\texttt{true}}\  \textbf{\texttt{||}}\ e
  \ \rightarrow \ 
  \textbf{\texttt{true}}
}
\]

\vspace{10mm}
\textbf{Or-shortcut-false}: An expression with binary operator
$\textbf{\texttt{||}}$ whose left sub-expression is
$\textbf{\texttt{false}}$ can be reduced to
the right sub-expression.
\[
\frac{
}{
  \textbf{\texttt{false}}\  \textbf{\texttt{||}}\ e
  \ \rightarrow \ 
  e
}
\]

\vspace{10mm}
\textbf{Right-binary-reduce}: An expression with binary operator
can be reduced if its left sub-expression is a value and its right
sub-expression can be reduced.
\[
\frac{
  e_2\ \rightarrow\ e_2', \textrm{and}\ \textit{binary-operator}
  \mbox{is not}\ \textbf{\texttt{\&\&}}\ \textrm{or}\ \texttt{\textbf{||}}
}{
  v\  \textit{binary-operator} \ e_2
  \ \rightarrow \ 
  v\  \textit{binary-operator} \ e_2'
}
\]

\vspace{10mm}
\textbf{Prim-binary-reduce}: An expression with binary operator
can be reduced if its left and right sub-expressions are values and
the corresponding function is defined for those values.
\[
\frac{
  v\ \mbox{is result of}\ v_1\  \textit{binary-operator} \ v_2
}{
  v_1\  \textit{binary-operator} \ v_2
  \ \rightarrow \ 
  v
}
\]

\subsection*{Expressions: Unary operators}

\textbf{Unary-reduce}: An expression with unary operator
can be reduced if its sub-expression can be reduced.
\[
\frac{
  e \ \rightarrow \ e'
}{
  \textit{unary-operator} \ e
  \ \rightarrow \ 
  \textit{unary-operator} \ e'
}
\]

\vspace{10mm}
\textbf{Prim-unary-reduce}: An expression with unary operator
can be reduced if its sub-expression is a value and
the corresponding function is defined for that value.
\[
\frac{
  v'\ \mbox{is result of}\ \textit{unary-operator} \ v
}{
  \textit{unary-operator} \ v
  \ \rightarrow \ 
  v'
}
\]

\subsection*{Expressions: conditionals}

\textbf{Conditional-predicate-reduce}: A conditional
expression can be reduced, if its predicate can be reduced.
\[
\frac{
  e_1 \ \rightarrow \ e_1'
}{
  e_1\  \textbf{\texttt{?}}\ e_2\ \textbf{\texttt{:}}\ e_3
  \ \rightarrow \ 
  e_1'\ \textbf{\texttt{?}}\ e_2\ \textbf{\texttt{:}}\ e_3
}
\]

\vspace{10mm}
\textbf{Conditional-true-reduce}: A conditional
expression whose predicate is the value
$\textbf{\texttt{true}}$
can be reduced to its consequent expression.
\[
\frac{
}{
  \textbf{\texttt{true}}\  \textbf{\texttt{?}}\ e_1\ \textbf{\texttt{:}}\ e_2
  \ \rightarrow \ 
  e_1
}
\]

\vspace{10mm}
\textbf{Conditional-false-reduce}: A conditional
expression whose predicate is the value
$\textbf{\texttt{false}}$
can be reduced to its alternative expression.
\[
\frac{
}{
  \textbf{\texttt{false}}\  \textbf{\texttt{?}}\ e_1\ \textbf{\texttt{:}}\ e_2
  \ \rightarrow \ 
  e_2
}
\]


\subsection*{Expressions: function application}

\textbf{Application-functor-reduce}: A function application
can be reduced if its functor expression can be reduced.
\[
\frac{
  e \ \rightarrow \ e'
}{
  e\  \textbf{\texttt{(}}\ \textit{expressions} \ \textbf{\texttt{)}}
  \ \rightarrow \ 
  e'\  \textbf{\texttt{(}}\ \textit{expressions} \ \textbf{\texttt{)}}
}
\]


\vspace{10mm}
\textbf{Application-argument-reduce}: A function application
can be reduced if one of its argument expressions can be reduced and all
preceding arguments are values.
\[
\frac{
  e \ \rightarrow \ e'
}{
  v\  \textbf{\texttt{(}}\ v_1 \ldots v_i \ e \ldots\ \textbf{\texttt{)}}
  \ \rightarrow \ 
  v\  \textbf{\texttt{(}}\ v_1 \ldots v_i \ e' \ldots\ \textbf{\texttt{)}}
}
\]



\vspace{10mm}
\textbf{Function-declaration-application-reduce}:
The application of a function declaration
can be reduced, if all
arguments are values. 
\[
\frac{
  f = \textbf{\texttt{function}}\  \textit{n} \ 
                 \textbf{\texttt{(}}\  x_1 \ldots x_n
                 \ \textbf{\texttt{)}}\ \textit{block}
}{
  f\ \textbf{\texttt{(}}\ v_1 \ldots v_n\ \textbf{\texttt{)}}
  \ \rightarrow \ 
  \textit{block} [x_1 \leftarrow v_1]\ldots[x_n \leftarrow v_n]
  [n \leftarrow f]
}
\]


    \end{document}

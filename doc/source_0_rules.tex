\section*{Reduction}

A \emph{thunk} is a pair whose first component is a program or expression
and whose
second component is an environment, and an 
\emph{environment} is a mapping from names to programs, all of which are
function declarations. The environment extension operation
$E [ n \leftarrow f]$ returns an environment that is extended by
a new binding: $E[n \leftarrow f](m) = f$ if $n = m$ and
$E[n \leftarrow f](m) = E(m)$ otherwise.

The \emph{reducer} $\Rightarrow$ is a partial function from thunks to thunks,
and $\Rightarrow^*$ is its reflexive transitive closure.
A \emph{reduction} is a sequence of thunks
$t_1 \Rightarrow^* t_n$, where $t_n$ is not reducible, i.e. there is no thunk
s such that $t_n \Rightarrow s$.

The result of evaluating a given program $p$ is a program or expression $r$
such that $(p, \emptyset) \Rightarrow^* (r, E)$ is a reduction,
where $\emptyset$ is the empty environment.

The statement $\epsilon$ is the empty statement, introduced for convenience.

A \emph{value} is a primitive number expresssion, primitive boolean expression
or a primitive string expression.
\begin{eqnarray*}
(\epsilon  \ \textit{statement} \ldots, E)
  & \Rightarrow &
  (\textit{statement} \ldots, E) \\
(\textit{statement} \ldots, E)
  & \Rightarrow &
  (\textit{statement}' \ldots, E)\\
                                       && \textrm{if}\
                                          (\textit{statement},E) \Rightarrow
                                          (\textit{statement}',E) \\
( f\ \textit{statement} \ldots, E)
  & \Rightarrow &
  ( \textit{statement} \ldots, E[\textit{name} \leftarrow f]) \\
                                       && \textrm{if}\
                                 f = \textbf{\texttt{function}}\  \textit{name} \ 
                                 \textbf{\texttt{(}}\  \textit{parameters}
                                 \ \textbf{\texttt{)}}\ \textit{block}    \\
(\textit{expression} \textbf{\texttt{;}} \ldots, E)
  & \Rightarrow &
  ( \textit{expression}'\textbf{\texttt{;}} \ldots, E) \\
                                       && \textrm{if}\
                                 (\textit{expression}, E) \Rightarrow
                                 (\textit{expression}', E)  \\
(v \textbf{\texttt{;}}\ \textit{statement}+, E)
  & \Rightarrow &
  ( \epsilon\ \textit{statement}+, E) \\
                                       && \textrm{if}\
                                 v \ \mbox{is a value}
\end{eqnarray*}


  %% && \textit{parameters}   && ::= &\quad &&  \epsilon\ | \  \textit{name} \ 
%%                                                    (\ \textbf{\texttt{,}} \ \textit{name}\ )\ \ldots
%%                                                             && \textrm{function parameters}   \\[1mm]
%% && \textit{block}        && ::= &      && \textbf{\texttt{\{}}\  \textbf{\texttt{return}}\ \textit{expression}   \ \textbf{\texttt{\}}} \quad
%%                                                            && \textrm{function body}\\[1mm]         
%% && \textit{expression}   && ::= &\quad &&  \textit{number}   && \textrm{primitive number expression}\\
%% &&                       && |   &\quad && \textbf{\texttt{true}}\ |\ \textbf{\texttt{false}}
%%                                                            && \textrm{primitive boolean expression}\\
%% &&                       && |   &\quad &&  \textit{string}   && \textrm{primitive string expression}\\
%% &&                       && |   &\quad &&  \textit{name}   && \textrm{name expression}\\
%% &&                       && |   &\quad &&  \textit{expression} \  \textit{binary-operator} \ 
%%                                             \textit{expression} \qquad
%%                                                            && \textrm{binary operator combination}\\
%% &&                       && |   &\quad &&   \textit{unary-operator} \ 
%%                                             \textit{expression}
%%                                                            && \textrm{unary operator combination}\\
%% &&                       && |   &\quad &&   \textit{expression} \ 
%%                                             \textbf{\texttt{(}}\ \textit{expressions}\
%%                                             \textbf{\texttt{)}}
%%                                                            && \textrm{function application}\\
%% &&                       && |   &\quad &&   \textit{expression} \ \textbf{\texttt{?}}\ 
%%                                             \textit{expression}
%%                                             \ \textbf{\texttt{:}}\
%%                                             \textit{expression}\
%%                                                            && \textrm{conditional expression}\\
%% &&                       && |   &\quad &&  \textbf{\texttt{(}}\  \textit{expression} \ 
%%                                             \textbf{\texttt{)}} && \textrm{parenthesised expression}\\[1mm]
%% && \textit{binary-operator}    \ 
%%                         && ::= &\quad && \textbf{\texttt{+}}\ |\ \textbf{\texttt{-}}\ |\ \textbf{\texttt{*}}\ |\ \textbf{\texttt{/}}\ |\ \textbf{\texttt{\%}}\ |\ 
%%                                    \textbf{\texttt{===}}\ |\ \textbf{\texttt{!==}}\ \\
%% &&                       && |  &\quad &&  \texttt{\textbf{>}}\ |\ \texttt{\textbf{<}}\ |\ \texttt{\textbf{>=}}\ |\ \texttt{\textbf{<=}}\
%%                                           |\ \textbf{\texttt{\&\&}}\ |\ \texttt{\textbf{||}}\  \\[1mm]
%% && \textit{unary-operator}    
%%                         && ::= &\quad && \textbf{\texttt{!}}\ |\ \textbf{\texttt{-}}\\[1mm]
%% && \textit{expressions}  && ::= &\quad && \epsilon\ | \ \textit{expression}\ (
%%                                                                \ \textbf{\texttt{,}} \
%%                                                                  \textit{expression} \ 
%%                                                                       )\ \ldots
%%                                                             && \textrm{argument expressions} 
%% \end{alignat*}

    \end{document}
